\graphicspath{{img/music_notation/}}

Music notation, when properly applied, can completely describe any musical score in a simple, concise manner. In order
to achieve this, music notation must describe all definable parameters of each sound, specifically\cite{mcgrain1990music}:

\begin{itemize}
    \item duration
    \item pitch
    \item dynamic
    \item timbre
\end{itemize}

\textbf{Duration} is described by time signature ($4\atop4$, $3\atop4$, $7\atop8$, etc.), tempo (primarily, beats per
minute: \Vier~=~120), and duration values of note-heads (\Ganz,~\Halb,~\Vier,~\AAcht,~\Sech,~etc.) and rests
(\GaPa,~\ViPa,~\AcPa,~\SePa,~etc.):

%\figcenter\lilypondfile{ly/music_notation/duration.ly}


\textbf{Pitch} is defined by position of the note on the staff, key, accidentals ($\flat$,~$\sharp$,~$\natural$) and
the specified clef (\inlinemusic\smalltrebleclef, \inlinemusic\smallbassclef, \inlinemusic\smallaltoclef, etc.):

%\figcenter\lilypondfile{ly/music_notation/pitch.ly}


\textbf{Dynamic} of a sound describes its amplitude or loudness (!! mf, p, ff), its emotional intensity and change
through time (crescendo, ..)

\textbf{Timbre} describes specific color of a played note/sound. Timber primarily depends on the instrument played but
also can define other instrumental directions (i.e. \textit{on bell of cymbal}, etc.)


% The Staff
\section{The Staff}\label{sec:the_staff}
The base for all musical scores is the \textit{staff}. All other music symbols go are placed on the staff or in relation
to it.

The staff consists of five horizontal lines and four spaces between the lines. Every note-head is placed on one of
the lines or on one of the spaces between the lines. The higher the note-head on the staff - the higher the pitch of
the produced note.

%\figcenter\lilypondfile{ly/music_notation/staff.ly}


\section{Leger Lines}\label{sec:leger_lines}
Obviously, five lines and five spaces can provide only limited range of notes (precisely, eleven places to put
the note-head, including just beneath the first(bottom) line and above fifth(top) line). If notes from outside this
range are needed, they are placed on or between so-called \textit{Leger lines}. These are the lines placed above or
beneath the main staff only in places where they are needed, so for each note individually.

%\figcenter\lilypondfile{ly/music_notation/leger_lines.ly}

% Clefs
\section{Clefs}\label{sec:clefs}

The specified \textit{clef} defines location of each pitch on the staff. The most commonly used clefs are the Treble and
the Bass clefs\cite{an-explanation-of-clefs}.

\subsection{The Treble Clef}\label{subsec:the-treble-clef}

The \textit{Treble Clef} (or \textit{G clef}, because the middle curl of it encircles line on the staff that represents a
G-note) is used for most high-sounding instruments (i.e.\ violin, guitar, ukulele, flute, clarinet, saxophone, trumpet,
etc.).

%\figcenter\lilypondfile{ly/music_notation/treble_clef.ly}

As it defines second line as G, the lines on the staff, from bottom to top, are E, G, B, D, F. The spaces then
are F, A, C, E\@. The middle C\footnote{\textit{Middle C} is a commonly used reference note. It is a closest C to
the middle of a standard 88 key piano (specifically, fourth C from the left). It is around 261.63 hertz.} goes on
the first leger line below the treble staff.

\subsection{The Bass Clef}\label{subsec:the-bass-clef}

The \textit{Base Clef} (or \textit{F clef}, because line between two dots on the symbol represents an F-note) is used for
low sounding instruments (i.e.\ bass guitar, cello, trombone, tuba, etc.)

%\figcenter\lilypondfile{ly/music_notation/bass_clef.ly}

As it defines fourth line as F, the lines on the staff are G, B, D, F, A, and the spaces are A, C, E, G\@. The middle C
goes on the first leger line above the bass clef.


\subsection{The Percussion Clef}\label{subsec:the-percussion-clef}

The \textit{Percussion Clef} is commonly used for drum-set notation. Each line and space represent different part of
the drum kit. They ar often predefined at the start of the part in so-called \textit{key} or \textit{legend}, or when
they first appear in the score.

%\figcenter\lilypondfile{ly/music_notation/percussion_clef.ly}


\subsection{The Alto and Tenor Clefs}\label{subsec:the-alto-and-tenor-clefs}

\textit{Alto Clef} (or \textit{C clef}, because line in the middle of the alto staff represent middle C) and
The \textit{Tenor Clef} are less often used clefs. The viola and the alto trombone are generally the only instruments that
use the Alto clef. Tenor clef is occasionally used to represent the upper ranges of the cello, double bass, bassoon,
and trombone.

%\figcenter\lilypondfile{ly/music_notation/alto_tenor_clefs.ly}
The lines of the alto staff are F, A, C, E, G, and the spaces are G, B, D, F\@. Similarly, for tenor clef, C is moved up
one line from alto clef, making the notes on the lines D, F, A, C, E and notes in the spaces E, G, B, D\@.

\subsection{The Great Staff}\label{subsec:the-great-staff}
The \textit{Great Staff} or the \textit{Grand Staff} is a combination of the treble staff and the bass staff. Usually
used by piano or harp musicians.

%\figcenter\lilypondfile{ly/music_notation/great_staff.ly}

Often they also divide score into to parts played by left and right hand (i.e.\ on piano, treble clef part with
the right hand, bass clef part with left hand). So, even if some notes belong to treble clef they may be put on leger
lines above bass clef if played by left hand and vice versa.

%\figcenter\lilypondfile{ly/music_notation/great_staff_leger_lines.ly}


% Rhythmic Description
\section{Rhythmic Description}\label{sec:rhythmic-description}
Alongside with pitch, it is required to describe rhythm. \textit{Rhythmic description} determines exactly when note
should be played and when it should stop playing. Notationally it is defined by note-heads, stems, flags, beams, rests,
and time signature.

\subsection{Note-heads, stems, flags, beams}\label{subsec:note-heads}

There are two types of note heads open and closed.
%\figcenter\lilypondfile{ly/music_notation/note_heads.ly}

\textit{Stems} are vertical lines attached to the side of the notes-head. Together with flags, beams, and augmentation
dots they define duration value:

%\figcenter\lilypondfile{ly/music_notation/notes_duration.ly}


\subsection{Rests}\label{subsec:rests}
Same as for notes, we can define pauses in music - \textit{rests}:

%\figcenter\lilypondfile{ly/music_notation/rests_duration.ly}


\subsection{Time signatures}\label{subsec:time-signatures}
Time signature is a sign that indicates the metre of a composition. Most time signatures consist of two vertically
aligned numbers, such as $2\atop2$, $3\atop4$, $6\atop8$, and $11\atop16$. The top figure reflects the number of beats
in each measure, or metrical unit; the bottom figure indicates the note value that receives one beat (here,
respectively, half note, quarter note, eighth note, and sixteenth note). When measures contain an uneven number of
beats falling regularly into two subgroups, the division may be indicated as, for instance, ${3+4}\atop4$ instead
of $7\atop4$\cite{time-signature}.

%\figcenter\lilypondfile{ly/music_notation/time-signatures.ly}

