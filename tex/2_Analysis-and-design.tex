\chapter{Analysis and design}\label{ch:analysis-and-design}

This chapter defines architecture of the chosen solution. It provides details of used approaches for music sources
separation and models used in it, pitch extraction and signal analysis, event detection, etc.

\section{Architecture}\label{sec:architecture}

The implementation of the system is separated into logical parts responsible for sound data streaming, music source
separation, pitch and events detection, transcription and score generation. Following diagram shows architecture of
the solution. Arrows represent data flow. Dotted arrows represent flow that is optional. If given parameters (like
tuning, tempo, time signature and key) are specified by user, they are not being estimated. Each rectangular block
represents logical module in implementation.

Detailed description of each component is in the dedicated section following the diagram on \cref{fig:architecture}.

\image[0.74]{architecture}{pdf}{Architecture of the implementation.}

\section{Audio streaming}\label{sec:audio-streaming}

This implementation directly works only with \ac{WAVE} (.wav/.wave). Any other format is converted to WAVE first,
then processed.

\subsection{WAVE format}\label{subsec:wave-format}
\textbf{WAVE} is an audio file format standard, developed by Microsoft and IBM, for storing an audio bitstream on PCs.
What's important for this thesis and implementation is that it stores data in chunks in \ac{LPCM} format. This format
allows to perform \ac{DFT} used in pitch extraction.

\subsection{Sampling rate}\label{subsec:sampling-rate}
\ac{LPCM} mentioned above stores sampled amplitude of recorded audio at specific sampling rate (frequency, measured in
Hz).

The most common sampling rate is 44.1 kHz, or 44100 samples per second. This is the standard for most consumer audio,
used for formats like CDs\cite{digital-audio-basics}.

The sampling rate determines the range of frequencies captured in digital audio. The lowest frequency a person can hear
is 20 Hz. The highest frequency humans can hear are in the range of 20.000 Hz, but only young people can hear such high
tones\cite{roots-of-modern-technology}. According to \textit{Nyquist Theorem}, a signal which has a Fourier transform
having only frequencies upto a certain maximum $f_m$, we can obtain the analog signal $f(t)$ from the sampled signal
$f'(t)$ by passing the sampled signal $f'(t)$ through a low pass filter provided that the sampling frequency $f_s$ is
more than twice the maximum frequency $f_m$ present in the signal i.e.\ , $f_s > 2f_m$\cite{signals-and-systems}. Hence,
having 44100 Hz sampling rate, we can reproduce and analyse frequencies up to 22050 Hz (assuming an ideal low pass
filter). Otherwise, if recorder has a sampling rate lower than $2\times$ the highest frequency (which was not cut off
by low pass filter) it causes the effect called \textit{aliasing}, which introduces unexpected sounds in the recording
that were not present in the original sound. If the sampling frequency is too low the frequency spectrum overlaps, and
becomes corrupted\cite{signals-and-systems}.

The implementation is able to process input sound with any sampling rate, though lower sampling limits processed
frequencies range to lower pitches.

\pagebreak

\section{Music source separation}\label{sec:music-source-separation}

First step of the sound processing is separation of the sound into source instruments (i.e.\ voice, guitar, piano,
etc.)

As was mentioned in the previous chapter, this implementation uses \textit{Spleeter} for separation of source
instruments. \textit{Spleeter} is a fast and state-of-the art music source separation tool with pre-trained
models\cite{spleeter2019}. Its implementation contains three pre-trained models:

\begin{itemize}
	\item vocals/accompaniment separation,
	\item 4 stems separation as in \ac{SiSeC}\cite{stter20182018} (vocals, bass, drums and other),
	\item 5 stems separation with an extra piano stem (vocals, bass, drums, piano and other). It is, to the author's
	knowledge, the first released model to perform such a separation.
\end{itemize}

Estimations for all the models is performed in a frequency domain of the sound. Meaning that sound data from time domain
is converted to frequency domain using \ac{FFT}, passed to the models described in section~\ref{subsec:music-source-separation:u-net-architecture}
about U-net architecture. Output of the model is separated tracks for each instrument and voice. To get sound of each
instrument and voice in time domain (as it would be represented in \ac{WAVE}), we would need to pass it through
\ac{IDFT}. But it is not necessary, as all the subsequent processing will be performed on the sound in frequency domain.

More about \ac{FFT} is in the following section~\ref{sec:pitch-extraction} about pitch extraction.

\section{Pitch extraction}\label{sec:pitch-extraction}
As was mentioned in \cref{sec:multi-pitch-detection} there are many approaches to pitch (and specifically
to multi-pitch) detection. The one that is presented in this theses utilizes a combination of ideas defined in works
of Matti P Ryynanen et al.\cite{ryynanen2005polyphonic}, Arnaud Dessein et al.\cite{dessein2010real} and Paris Smaragdis
et al.\cite{smaragdis2003non}. Solution is joint, thus estimates played notes at a given moment all at once (opposed to
iterative approaches). It attempts to detect frequencies similarly to matrix factorization techniques through analysis
of sound spectrogram. But instead of matrix factorization (\ac{NMF}), this work attempts to detect notes' events,
specifically their envelopes (more on the sound envelope in \cref{subsec:sound-envelope}), using \ac{ML} models.
Specification of the used data and training of the models is in the \cref{subsec:data-and-model-training}.

\pagebreak

\subsection{Sound envelope}\label{subsec:sound-envelope}
\textit{Sound envelope} is a variation of the sound volume in time\cite{dregvaite2015information}. Sound envelope
consists of 4 stages: \ac{ADSR}:

\image{sound-envelope}{pdf}{Sound envelope.}

\cref{fig:sound-envelope} shows a theoretical simple \ac{ADSR} model of sound envelope. But different instruments
produce different envelopes depending on a nature of sound extraction:

\image{instrument-envelopes}{png}{Sound envelopes of piano and violin\cite{sound-envelope}.}

As seen on the \cref{fig:instrument-envelopes}, piano and any other instrument that produces sound by hitting, tapping
or pinching of a string (like guitar, harp, bandura, balalaika, etc.), will produce similar envelope with defined
attack (the moment of piano key pressing; pinching or hitting a string on guitar, etc.), decay and sustain (when piano
key remains pressed or piano sustain pedal is used, etc.) and release (when piano key and sustain pedal are released,
guitar strings are muted, etc.).


\subsection{Multiresolution \ac{FFT}}\label{subsec:multiresolution-fft}
https://pdfs.semanticscholar.org/d55f/984d569786e1bbf945f7683361ffbbfff79a.pdf

\section{Event detection}\label{sec:event-detection}

Note and subsequently its pitch and start are estimated by detecting its envelope. Having a sample of sound (change of
volume of each pitch as determined from \ac{FFT}) of duration $k$ seconds specified by implementation, predictive model
attempts to estimate whether note has been played at a given point in time by detecting its envelope that should look
similar for each note of the given instrument. That means that there will be a model for each predefined instrument
trained on its samples (more in \cref{subsec:data-and-model-training}).

\subsection{Data and model training}\label{subsec:data-and-model-training}
Dataset for training of the above-mentioned models was generated from the \textit{NSynth dataset}\cite{nsynth2017}.
\textit{NSynth} is an audio dataset containing 305,979 musical notes, each with a unique pitch, timbre, and envelope.
For 1,006 instruments from commercial sample libraries, there are generated four second, monophonic 16kHz audio
snippets, referred to as notes, by ranging over every pitch of a standard MIDI piano (21-108) as well as five different
velocities (25, 50, 75, 100, 127)\cite{nsynth2017}.

NSynth contains samples for 11 different instruments: bass, brass, flute, guitar, keyboard, mallet, organ, reed,
string, synth\_lead, vocal. They are all stored in \ac{WAVE} format and have needed metadata in JSON format alongside
with them. Metadata for each sample includes instrument, note, pitch and velocity in \ac{MIDI} format (in the range
[0, 127]), and sampling rate.

Spleeter, used for source separation, is able to separate sound only into 5 source instruments. Hence only those samples
from NSynth will be used to generate models.

The way, the training dataset is generated, is completely determined by the way, the input stream is processed during
transcription. Preprocessing of input stream and training data will be the same. The whole data flow is shown
on \cref{fig:pitch-detection}.

\image[1.15]{pitch-detection}{pdf}{Data flow for pitch detection.}

As shown on \cref{fig:pitch-detection}, input stream (blue) is a stream of data read from input file or microphone (or
any other input). It is read by chunks of size $c$ determined by implementation. Each window of $k$ chunks is passed
through \ac{FFT} to transform data to a frequency domain. Taking several chunks of data to pass it through \ac{FFT}
increases its accuracy, peaks of played pitches become more prominent and output becomes more robust to noise and phase
shifts. Overlapping of $k$-sized windows allows producing more data-points per second and subsequently features for
predictive model. Having $T$ chunks and window of size $k$, produced spectrogram is of size $F,T-k$ where $F$ is
a number of detected frequencies.

After transforming input to time-frequency spectrogram (red), frequencies are translated to musical notes (green).
Assuming equal temperament tuning and \textit{A} with 440Hz frequency (actual tinning will be estimated later in
the analysis), frequencies are converted to the closest note. Frequencies converted to the same note are filtered
to leave only the highest volume value.

The output of previous step is passed to the model of a given instrument by window of size $m$. So $m$ is a number of
input features of the model. The model attempts to classify whether given window contains an envelope of a played sound
that starts from a given point in time. So for $\forall{i} \in [0,N], j \in [0,T-k-m]$, $p_{i,j}$ (pink) shows
prediction of the model for note $i$ at a time $j$.

Training data is generated from NSynth dataset in a similar fashion. Positive labels are set for the pitch being played
in a sample, negative for all the others. Also negative examples are generated from the same sample for played pitch but
with a time shift, starting the example from or ending it somewhere on the middle of the actual sound envelope.

\subsection{Post-processing}\label{subsec:post-processing}
Duration of the note is determine by its start (start of the sample passed to the model) and its end (moment, when
note's volume lowers under the predefined threshold). As was mentioned in \cref{ch:state-of-the-art}, implementation
also utilizes several simple postprocessing ideas:
\begin{itemize}
	\item if the duration of the note is too short, it is omitted,
	\item if the duration between end and next start of the same note is too short, it is combine into a single note,
	\item if the note played at the same time with another note but with the significant difference in volume,
	the quieter note is omitted.
\end{itemize}


\section{Tuning classification}\label{sec:tunning-classification}
Tuning of the instrument is not a part of score transcribed into sheet music and most of the Western music follows
the same tuning system. Specifically, equal temperament system with \textit{A} tuned to 440Hz. But tuning estimation is
an essential part for correct reproduction of the sound.

There are two primary parts to tuning estimation: detection of frequency of base note, commonly \textit{A}:
\figcenter{
\lilypondfile{ly/analysis_and_design/base_a.ly}
\caption{Base \textit{A} commonly tuned to 440Hz.}
}
and tuning system, like equal temperament, pythagorean, meantone, etc. While tuning system in the Western music very
rarely diverges from equal temperament, frequency of a base note can often be chosen to be different from 440Hz. That
also often might happen for instruments that are often being retuned like guitars as tuning ``by ear'' by person that
does not have a perfect pitch\footnote{\textit{Perfect pitch} or \textit{absolute pitch} is the ability to identify
a note by hearing it. The ability is considered remarkably rare, estimated to be less than one in 10,000
individuals\cite{perfect-pitch}.} is defined by tuning of the string relative to which all other strings are tuned.

Having found one played frequency $F$ in sound (or several frequencies for better precision), it is matched to
the closest note $N$ in 440Hz-A-tuning, then real frequency of \textit{A} note $f(A)$ is calculated as:

\[ f(A) = F * 2^\frac{t(N)-t(A)}{12} \]

where $t(N)$ is an index number of the note(semitone) $N$ (for example in \ac{MIDI} representation). Now, all the other
notes can be calculated in the same way.

\section{Tempo estimation}\label{sec:tempo-estimation}
As was mentioned in \cref{subsec:tempo-and-time-signature-estimation}, \textit{Madmom} library will be used for
the task of tempo estimation. Authors use a recurrent neural network to learn an intermediate beat-level representation
of the audio signal. The output of the neural network is a beat activation function, which represents the probability of
a frame being a beat position. And instead of processing the beat activation function to extract the positions of
the beats, authors use it directly as a one-dimensional input to the bank of resonating comb filters. \text{Comb
filtering} can be defined as ``the frequency response caused by combining a sound with its delayed duplicate.
The frequency response displays a series of peaks and dips caused by phase interference. The peaks and dips look like
the teeth in a comb, with very narrow, deep notches where signals are attenuated.''\cite{comb-filter}. Using comb
filters with different lags (delays) implementation of madmom detects at which lag the beat of the sound resonates
the most. Given lag would then define a tempo.

The range of possible tempo values (\ac{BPM}) $t$ is limited to $1 \le t \le 128$ and to only whole numbers. This is
decided so that the range can include loops that last from only 1 beat to 128 beats, which would correspond to a maximum
of 32 bars in $4\atop4$ meter.

\section{Time signature estimation}\label{sec:time-signature-estimation}
Problem of time signature or \textit{meter} estimation is similar to tempo detection in a sense that the basic idea of
it is finding its repetitiveness, recurrence - beat for tempo and content of a sheet music bar for time signature. That
is why solutions for these problems often overlap.

Another approach for tempo as well as for time signature estimation is autocorrelation modeling. Autocorrelation
modeling is used to determine the length of the bar - number of beats per each meter. Technically, time signature
definition can contain any numbers for number of beats per measure (top number) and the note value that receives one
beat (bottom number). But most of music peaces of western music use powers of 2 as a note value (otherwise it is called
irrational measure) and rarely higher than 8 (\AAcht). As for number of beats per bar, a 1 or values higher than 12 are
considered to be the unusual time signatures. So the implementation limits estimation to these ranges.

Knowing the tempo - a number of fourth notes (\Vier) per minute, taking an average volume of the notes (0 if no notes
are there) in all position in time of sixteenth notes(\Sech) produces the time series on which implementation models
autocorrelation. Assuming that rhythmical structure of the bar and position of strong and weak\footnote{Commonly, some
notes per bar are \textit{strong}(louder) and some are \textit{weak}(quieter). This determines accents in measure. For
example in $4\atop4$ time, first beat is often the loudest(strong), third is also strong, but not as strong as
the first, and second and fourth are weak.} beats is continues through the whole peace or its significant part, the lag
of modeled autocorrelation will determine the number of sixteenth notes per bar.

\ac{ARIMA} model is used to determine the lag. \ac{ARIMA} is a class of models that ``explains'' a given time series
based on its own past values, that is, its own lags (AR part) and the lagged forecast errors (MA part), so that equation
can be used to forecast future values. Specifically its simpler version AR will be used. AR is defined as follows:
\[ Y_t = \alpha + \beta_1*Y_{t-1} + \beta_2*Y_{t-2} + \dots + \beta_n*Y_{t-n} + \epsilon_1 \]
where $Y_t$ is value measured in time $t$, $\alpha$ is the intercept term, $\beta_k$ is coefficient of the first lag,
and $\epsilon$ is a noise. All of $\beta$s and $\alpha$ are estimated by the model. The higher the absolute value
$\beta_k$ - the higher the correlation between the signal and its copy delayed on lag $k$. Obviously the highest
correlation of a signal is with its 0 lag. But as was mentioned above, the choice is limited to range $2 \le k \le 12$
with 8 as a shortest note value. So the coefficient $\beta$ are estimated from 8th up to 32nd lag to determine time
signatures from $2\atop4$ (which in time is equivalent to $8\atop16$) up to $12\atop8$ (which in time is equivalent to
$24\atop16$) and up to $8\atop4$ or $4\atop2$ (which in time are equivalent to $32\atop16$).

Technically any score for which appropriate $k$ was found, can be written within $k\atop16$ measure. But it is better to
identify the best logical value for the number of beats per measure for simplification of reading of the score and
convert the time signature to either $k/2\atop8$, $k/4\atop4$, or $k/8\atop2$.

Having found a value of $k$, converting is performed according to a \cref{fig:time-signatures-table} as indicated by
green cells for even values of $k$. Odd values of $k$ are not expected as they are very unusual in measures of
$k\atop16$. If they are odd time signature is left as is. Important to note that measures indicated by green cell are
more common than their counterparts within a row but time signatures like $6\atop8$ and $2\atop2$ are also widely used
in music. But it is hard to objectively identify which of the measures $6\atop8$ or $3\atop4$, $2\atop2$ or $4\atop4$
should be used.

\begin{table}[]
	\begin{center}
		\def\arraystretch{1.3}
		\begin{tabular}{|c||c|c|c|c|}
			\hline
			 k  & 16 (\Sech)   & 8 (\AAcht)   & 4 (\Vier)   & 2 (\Halb)  \\ \hline
			 8  & $8\atop16$   & $4\atop8$    & \cellcolor[HTML]{9AFF99}$2\atop4$   & $1\atop2$  \\ \hline
			 10 & $10\atop16$  & \cellcolor[HTML]{9AFF99}$5\atop8$    &             &            \\ \hline
			 12 & $12\atop16$  & $6\atop8$    & \cellcolor[HTML]{9AFF99}$3\atop4$   &            \\ \hline
			 14 & $14\atop16$  & \cellcolor[HTML]{9AFF99}$7\atop8$    &             &            \\ \hline
			 16 & $16\atop16$  & $8\atop8$    & \cellcolor[HTML]{9AFF99}$4\atop4$   & $2\atop2$  \\ \hline
			 18 & $18\atop16$  & \cellcolor[HTML]{9AFF99}$9\atop8$    &             &            \\ \hline
			 20 & $20\atop16$  & $10\atop8$   & \cellcolor[HTML]{9AFF99}$5\atop4$   &            \\ \hline
			 22 & $22\atop16$  & \cellcolor[HTML]{9AFF99}$11\atop8$   &             &            \\ \hline
			 24 & $24\atop16$  & $12\atop8$   & \cellcolor[HTML]{9AFF99}$6\atop4$   & $3\atop2$  \\ \hline
			 26 & $26\atop16$  & \cellcolor[HTML]{9AFF99}$13\atop8$   &             &            \\ \hline
			 28 & $28\atop16$  & $14\atop8$   & \cellcolor[HTML]{9AFF99}$7\atop4$   &            \\ \hline
			 30 & $30\atop16$  & \cellcolor[HTML]{9AFF99}$15\atop8$   &             &            \\ \hline
			 32 & $32\atop16$  & $16\atop8$   & $8\atop4$   & \cellcolor[HTML]{9AFF99}$4\atop2$  \\ \hline
		\end{tabular}
		\caption{Time signature selection table.}
		\label{fig:time-signatures-table}
	\end{center}
\end{table}


\section{Key classification}\label{sec:key-classification}
Key signature is a part of sheet music notation that simplifies it by avoiding redundant repetitive accidentals (sharps
and flats) and by defining the set of used notes which in its turn defines the set of used chords, their progressions
and harmonic functions in a piece of music. Detailed description of key signature is in a \cref{subsec:key-signature}.

As was mentioned in \cref{subsec:key-and-chord-detection}, there are several approaches to key classification including
template matching or \acp{HMM}. The best paid solution for key detection is \textit{Mixed In Key}\cite{mixed-in-keys}
having 95\% accuracy on their test dataset. The best free solution is \textit{KeyFinder}\cite{key-finder} with accuracy
of 77\%. But it is implemented in C++ so is hard to incorporated into Python implementation used in this work.

A simple heuristical solution was used in the framework of this thesis. For each note that has a sharp counterpart
(\textit{C}, \textit{D}, \textit{F}, \textit{G}, and \textit{A}), if its semitone-higher (sharper) note appears more
often than its natural note, its sharp symbol is included into a key signature.


\section{Post processing}\label{sec:post-processing}

\section{Transcription}\label{sec:transcription}

\section{Score generation}\label{sec:score-generation}



