\chapter{State-of-the-art}\label{ch:state-of-the-art}

This chapter discusses existing state-of-the-art solutions for music transcription.

Sound transcription into sheet music is a combination of several techniques that include but not limited to source
instruments separation, pitch/note detection, event detection, etc.

Source separation and sound transcription to sheet music are fairly independent processes so their description and
approaches may come from different sources and different projects. Therefore, implementation will also be separated.

\section{Source separation}\label{sec:source-separation}

There were many successful attempts for music score source separation\cite{spleeter2019,singing-voice-separation,singing-voice-separation-article}.
Performance of such projects are commonly measured according to \textit{\ac{SiSeC}}\cite{stter20182018}
on the standard \textit{musdb18}\cite{musdb18} and \textit{DSD100}\cite{SiSEC16} datasets.

Latest and most successful project in this field is \textit{Spleeter}\cite{spleeter2019}. It is a project of
Deezer\footnote{Deezer is a French online music streaming service (deezer.com).}. It takes similar approaches to previous solutions
by University of London and Spotify\cite{singing-voice-separation}. Spleeter's pre-trained models will be used in the
module responsible for music source separation described in detail in the chapters~\ref{ch:analysis-and-design} and~\ref{ch:implementation}.

Following approaches are described in~\cite{spleeter2019,singing-voice-separation,singing-voice-separation-article}.

\subsection{Spleeters approach}\label{subsec:music-source-separation:approach}
The pre-trained models are U-nets\cite{singing-voice-separation} and follow similar specifications as in
\textit{Singing voice separation: a study on training data}\cite{singing-voice-separation-article}. The U-net is a
encoder/decoder \ac{CNN} architecture with skip connections\cite{spleeter2019}. Architecture used in this approach
showed a state-of-the-art results on DSD100 dataset\cite{singing-voice-separation} and in the last \ac{SiSeC}\cite{SiSEC16}.

\subsection{U-net architecture}\label{subsec:music-source-separation:u-net-architecture}
The U-Net shares the same architecture (shown on \cref{fig:u-net-architecture}) as a convolutional autoencoder with
extra skip-connections that bring back detailed information lost during the encoding stage to the decoding stage. It has
five strided\footnote{Transposed convolutions – also called \textit{fractionally strided convolutions} – work by
swapping the forward and backward passes of a convolution. One way to put it is to note that the kernel defines
a convolution, but whether it’s a direct convolution or a transposed convolution is determined by how the forward and
backward passes are computed.\cite{dumoulin2016guide}} 2D convolution layers in the encoder and five strided 2D
deconvolution layers in the decoder.

The goal of the neural network architecture is to predict the vocal and instrumental components of its input indirectly:
the output of the final decoder layer is a soft mask for each source that is multiplied element-wise with the mixed
spectrogram to obtain the final estimate.

\image{u-net-architecture}{pdf}{Network Architecture\cite{singing-voice-separation}}

\pagebreak

\subsection{Data and training}\label{subsec:music-source-separation:data-and-training}

Spleeter's training dataset is an internal Deezer's dataset and is not shared (for copyright reasons).

Another project with similar approach, as explained in the dedicated article\cite{singing-voice-separation-article},
uses two datasets during training of the models: \textit{MUSDB}\cite{musdb18} and \textit{Bean}(private dataset).

\textit{MUSDB} is the largest and most up-to-date public dataset for source separation\cite{musdb18}. It contains 150
songs of western music genres primarily pop/rock, some hip-hop, rap and metal songs. And each song consists of four
audio tracks: drums, bass, vocal and other. Original mix (and input of the model) is produced by summing tracks of four
sources (expected outputs) together.


\section{Multi-pitch Detection}\label{sec:multi-pitch-detection}
There were several projects utilizing different approaches to a problem of \ac{AMT}. Following section discuss these
approaches and projects that used them.

The most important part of transcription of sound into sheet music is pitch (and subsequently note) detection. The core
problem of polyphonic music transcription is multi-pitch detection.

In his Ph.D.\ research \textit{Multiple fundamental frequency estimation of polyphonic recordings}\cite{fundamental-frequency-estimation},
Chunghsin Yeh classifies multi-pitch detection systems according to their estimation type into two categories: joint and
iterative. The \textit{iterative} approach extracts the most eminent frequency per each iteration, until no other pitch
can be estimated and extracted. Commonly, iterative estimators generate errors on each iteration but are much cheaper in
terms of computation costs.

On the other side, the \textit{joint} estimation models evaluate combinations of pitches at once, which leads to
increase in accuracy but also in computation costs. Most of the latest approaches and state-of-the-art solutions fall
into joint category. Solution in this thesis also follows this category and will be discussed in detail in \cref{ch:analysis-and-design}.

\subsection{Feature-based multi-pitch detection}\label{subsec:feature-based-multi-pitch-detection}
Most multi-pitch estimation and note tracking approaches exploit methods that come from signal processing. There is no
specific model (\ac{ML} or other), and notes are detected using audio features that come from the input time-frequency
representation (spectrogram) either in an iterative or joint way. Usually, multi-pitch estimation uses a \textit{pitch
candidate set score function} or a \textit{pitch salience function}.

A \textit{salience function} is a function that provides an estimation of the predominance of different frequencies in
an audio signal at every time frame\cite{pitch-salience-function}. A \textit{pitch candidate set score function} is
function designed to evaluate the plausibility of the combination of the hypothetical sources\cite{fundamental-frequency-estimation}.

These feature-based techniques have produced the best results in the \ac{MIREX} multi-pitch and note tracking
evaluations. The work by Chunghsin Yeh\cite{fundamental-frequency-estimation} was the best performing method in
the \ac{MIREX} multi-pitch and note tracking tasks. Yeh proposed a joint pitch estimation algorithm based on a pitch
candidate set score function. Having a set of pitch candidates, the overlapping partials are detected and smoothed
according to the spectral smoothness principle, which states that the spectral envelope\footnote{\textit{Spectral
envelope} of the sound determines the particular vowel sound produced, and is, in general, one of the important acoustic
features that determine its perceived timbre\cite{kumar2007hierarchical}.} of a musical sound tends to be slowly
changing as a function of frequency. The score function for the pitch candidate set consists of four features:
harmonicity, mean bandwidth, spectral centroid, and ``synchronicity'' (synchrony). A polyphony inference mechanism based
on the score function increase selects the optimal pitch candidate set\cite{fundamental-frequency-estimation}.

In the following year, the best performing method for the \ac{MIREX} multi-pitch estimation and note tracking tasks,
Karin Dressler described in her work \textit{Multiple fundamental frequency extraction for MIREX}\cite{dressler2012multiple}.
A multiresolution \ac{FFT} (see \cref{subsec:multiresolution-fft} on multiresolution \ac{FFT}) analysis was used as
an input time/frequency representation, where the magnitude for each spectral bin was multiplied with the bin’s
instantaneous frequency. Pitch estimation is made by identifying spectral peaks and performing pair-wise analysis on
them, resulting on ranked peaks according to harmonicity, smoothness, the appearance of intermediate peaks, and harmonic
number. Finally, the system tracks tones over time using an adaptive magnitude and a harmonic magnitude threshold.

Other notable feature-based \ac{AMT} solution was introduced in the work by Pertusa and Inesta \textit{Multiple
fundamental frequency estimation using Gaussian smoothness and short context}\cite{pertusa2008multiple}. They proposed
a computationally inexpensive method for multi-pitch detection which computes a pitch salience function and evaluates
combinations of pitch candidates using a measure of distance between a \ac{HPS} and a smoothed \ac{HPS}. Another
approach for feature-based \ac{AMT} was proposed in \textit{Hybrid genetic algorithm based on gene fragment competition
for polyphonic music transcription}\cite{reis2008hybrid}, which uses genetic algorithms for estimating a transcription
by mutating the solution until it matches a similarity criterion between the original signal and the synthesized
transcribed signal.

More recently, Peter Grosche et al. proposed\cite{grosche2012automatic} an \ac{AMT} method based on a mid-level
representation derived from a multiresolution \ac{FFT} combined with an instantaneous frequency estimation. His system
also combines event (specifically start of the note) detection and tuning estimation for computing predictions. Finally,
Juhan Nam et al. proposed\cite{nam2011classification} a classification-based approach for piano transcription using
features learned from deep belief networks\cite{humphrey2013feature} for computing a mid-level time-pitch representation.

\subsection{Statistical model-based multi-pitch detection}\label{subsec:statistical-model-based-multi-pitch-detection}
Many approaches in the literature formulate the multi-pitch estimation problem within a statistical framework. As
Valentin Emiya et al. explains in their article \textit{Multipitch Estimation of Piano Sounds Using a New Probabilistic
Spectral Smoothness Principle}\cite{proba-spectral-smoothness}: given an observed frame $\pmb{x}$ and a set $\pmb{C}$ of
all possible fundamental frequency combinations, the frame-based multi-pitch estimation problem can then be viewed as
a \ac{MAP} estimation problem:

\[ \hat{C}_{MAP} = \argmax_{C \in \pmb{C}} P(C|\pmb{x}) = \argmax_{C \in \pmb{C}} \frac{P(\pmb{x}|C)P(C)}{P(\pmb{x})} \]

where $C = \{F_0^1, \dots, F_0^N\}$ is a set of possible frequencies (considering tuning of an instrument), $\pmb{C}$ is
the set of all possible $F_0$ combinations, and $\pmb{x}$ is the observed audio signal within a single analysis frame.

An example of \ac{MAP} estimation-based transcription is the \textit{PreFEst} system introduced by Masataka Goto in his
article \textit{A real-time music-scene-description system: predominant-F0 estimation for detecting melody and bass
lines in real-world audio signals}\cite{predominant-f0-estimation}, where each harmonic is modelled by a Gaussian
centered at its position on the log-frequency axis. \ac{EM} algorithm is used to estimate \ac{MAP} value.
An extension of this method was proposed by Kameoka et al. in \textit{A Multipitch Analyzer Based on Harmonic Temporal
Structured Clustering}\cite{harmonic-temporal-structured-clustering}, which jointly estimates multiple possible
frequencies, moments of start and end of the note, and dynamics. Partials are modelled using Gaussians placed at
the positions of partials in the log-frequency domain and the synchronous evolution of partials belonging to the same
source is modelled by Gaussian mixtures.

More recently, Peeling and Godsill, in their article \textit{Multiple pitch estimation using non-homogeneous Poisson
processes}\cite{peeling2011multiple}, also proposed a likelihood function for multiple-pitch estimation where for a
given time frame, the occurrence of peaks in the frequency domain is assumed to follow an inhomogeneous Poisson process.
Also, Koretz and Tabrikian in \textit{Maximum a posteriori probability multiple-pitch tracking using the harmonic
model}\cite{koretz2011maximum}, proposed an iterative method for multi-pitch estimation, which combines \ac{MAP} and
\ac{ML} criteria. The predominant source is expressed using a harmonic model while the remaining harmonic signals are
modelled as Gaussian interference sources\cite{koretz2011maximum}.

\pagebreak

\subsection{Spectrogram factorisation-based multi-pitch detection}\label{subsec:spectrogram-factorisation-based-multi-pitch-detection}
The majority of recent multi-pitch detection papers utilise and expand spectrogram factorisation techniques. \ac{NMF} is
a technique first introduced in their paper by Paris Smaragdis et al.\ \cite{smaragdis2003non} as a tool for music
transcription. In its simplest form, the \ac{NMF} model decomposes an input spectrogram
$\pmb{X} \in \mathbb{R}^{K \times N}_+$ with $K$ frequency bins and $N$ frames: \[ \pmb{X} \approx \pmb{WH} \]
where $R \ll K, N$; $\pmb{W} \in \mathbb{R}^{K \times R}_+$ contains the spectral bases for each of the $R$ pitch
components; and $\pmb{H} \in \mathbb{R}^{R \times N}_+$ is the pitch activity matrix across time.

In his paper \textit{Realtime multiple pitch observation using sparse non-negative constraints}, Cont Arshia applies
\ac{NMF} to \ac{AMT} problem. Sparseness constraints were added into the \ac{NMF} update rules, in order to find
meaningful transcriptions using a minimum number of non-zero elements in $\pmb{H}$. Emmanuel Vincent et al. in their
article \textit{Adaptive harmonic spectral decomposition for multiple pitch estimation}\cite{vincent2009adaptive}
incorporated harmonicity constraints in the \ac{NMF} model, resulting in two algorithms: harmonic and inharmonic
\ac{NMF}. The inharmonic version of the algorithm is also able to support deviations from perfect harmonicity and
standard equal temperament tuning. Also, Nancy Bertin et al. in their article\cite{bertin2010enforcing} proposed
a Bayesian framework for \ac{NMF}, which considers each pitch as a model of Gaussian components in harmonic positions.

More recently, Ochiai et al. in his paper \textit{Explicit beat structure modeling for non-negative matrix
factorization-based multipitch analysis}\cite{ochiai2012explicit} proposed an algorithm for multi-pitch detection and
beat structure analysis. The \ac{NMF} objective function is constrained using information from the rhythmic structure of
the recording. It helped to improve transcription accuracy in highly repetitive recordings.

This thesis approaches \ac{AMT} problem in similar fashion to spectrogram factorisation and feature-based methods.
Detailed description of used methods is in \cref{ch:analysis-and-design}.

\section{Note Tracking}\label{sec:note-tracking}

Typically \ac{AMT} algorithms compute a time-pitch representation which needs to be further processed in order to detect
note events with a discrete pitch value, a time of start and end of the note. This process is called \textit{note
tracking}. Most spectrogram factorisation-based methods estimate the binary piano-roll representation from the pitch
activation matrix using simple thresholding (i.e.\ in \textit{Explicit beat structure modeling for non-negative matrix
factorization-based multipitch analysis}\cite{grindlay2011transcribing} by Graham Grindlay and in \textit{Adaptive
harmonic spectral decomposition for multiple pitch estimation}\cite{vincent2009adaptive} by Emmanuel Vincent). This
approach will be used in the implementation of the thesis. Also, one simple and fast optimisation for note tracking is
minimum duration pruning, which is applied after thresholding (idea comes from paper by Arnaud Dessein et al.
\textit{Real-time polyphonic music transcription with non-negative matrix factorization and beta-divergence}\cite{dessein2010real}).
Primary idea is that output notes that have a duration smaller than a predefined threshold are removed from the final
score. Similar method was also used by Juan Pablo Bello et al. in their paper \textit{Automatic piano transcription
using frequency and time-domain information}\cite{bello2006automatic}, where more complex rules for note tracking were
used, addressing cases such as where a small gap exists between two note events. This method will also be applied as it
is easy to implement.

For threshold based method, there are several issues that may appear for different instruments, primarily related to how
notes are used and written for them in practice. For instance, for percussion instruments, note decay is exponential and
physical duration of the note is irrelevant as it is not controlled(for most percussion instruments) by a musician. so
notes may appear short and require pauses(rests) after them, even though they (rests) would not be written in sheet
music. Such problems may be solved with other rule based solutions specific to each instrument that requires them or
more complex approaches. Even though, in frameworks of this thesis, simple threshold-based solution was used for
the note tracking task, it is important to know other approaches for this problem.

There were several more complex solutions applied to the problem of note tracking. Following are some of them, though
without detailed description of the approach.

\acp{HMM} are frequently used at a stage of postprocessing of note tracking. In his work \textit{A discriminative model
for polyphonic piano transcription}\cite{poliner2006discriminative}, Graham Poliner proposes a note tracking method that
utilizes pitch-wise \acp{HMM}, where each \ac{HMM} has two states, indicating note activity and inactivity. \ac{HMM}
parameters (state transitions and priors) were learned directly from a ground-truth training set, while the observation
probability is given by the posteriogram output for a specific pitch.

In \textit{Polyphonic music transcription using note event modeling}\cite{ryynanen2005polyphonic} by Matti P Ryynanen
and Anssi Klapuri, a feature-based multi-pitch detection system was combined with a musicological model for estimating
musical key and note transition probabilities. Note events are described using 3-state \acp{HMM}, which model
the envelope (attack, sustain, and noise/silence states) of each sound. In addition, context-dependent \acp{HMM} were
employed in \textit{Automatic transcription of recorded music}\cite{grosche2012automatic} for determining note events by
combining the output of a multi-pitch detection system with an note-start detection system.

Finally, \acp{DBN} were proposed by Shigeki Sagayama et al. in their paper\cite{raczynski2009note} for note tracking.
They used the pitch activation of an \ac{NMF}-based multi-pitch detection algorithm as input. The \ac{DBN} has a note
layer in the lowest level, followed by a note combination layer. Model parameters were learned using MIDI files from
F. Chopin piano pieces.

\section{Tuning, time signature, key, and tempo estimation}\label{sec:other-amt-subtasks}
There are several other subtasks of \ac{AMT} systems that have to be resolved to be able to generate correct
transcription in a form of sheet music. Also, such estimates, if properly incorporated to the system, may improve
estimates of detected pitches and their durations, events. Tuning, time signature, key, and tempo estimation are such
tasks.

\subsection{Key and chord detection}\label{subsec:key-and-chord-detection}
Most Western music has a harmonic organisation around one key. The key is generally unchanged over whole, or at least
sections of musical pieces. At one point in time, the harmony may be described by chord, which is a combinations of
simultaneous or sequential notes which are perceived to belong together (and in general sound nice together, even though
any combination of notes has its chord). Algorithms for key (and similarly for chord) detection use template matching or
\acp{HMM}. For key detection, this theses uses the simple approach defined in \cref{sec:key-classification}.

\subsection{Tempo and time signature estimation}\label{subsec:tempo-and-time-signature-estimation}
The \textit{beats} are regularly spaced in time pulses. They are the primary unit that defines tempo and rhythm of most
Western music. A number of beats per unit of time (commonly per minute in sheet music) define \textit{tempo}. A number
of beats per uniform repetitive sections in score (bars) define \textit{time signature}. In order to interpret an audio
recording in terms of such a structure (which is necessary in order to produce Western music notation), the first step
is to determine the rate of the most salient pulse, which is the tempo.

Algorithms used for tempo induction include autocorrelation, Fourier transforms, and periodicity transform, which are
applied to audio features such as a note-start detection function (as Fabien Gouyon and Simon Dixon describe in their
article \textit{A review of automatic rhythm description systems}\cite{gouyon2005review}). The next step involves
estimating the timing of the beats constituting the main pulse, a task known as beat tracking. Again, numerous
approaches have been proposed, such as rule-based methods (as in \textit{Computational models of beat induction:
The rule-based approach}\cite{desain1999computational} by Peter Desain and Henkjan Honing), adaptive oscillators (as
in \textit{Resonance and the perception of musical meter}\cite{large1994resonance} by Edward W Large and John F Kolen),
agent-based or multiple hypothesis trackers (as in \textit{Automatic extraction of tempo and beat from expressive
performances}\cite{dixon2001automatic} by Simon Dixon), and other. The final step for rhythmic analysis consists of
estimating the time signature, which indicates how beats are grouped and subdivided at respectively higher and lower
metrical levels, and assigning each note-start and offset time to a position in this structure\cite{cemgil2011monte}.

