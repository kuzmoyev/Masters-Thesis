\chapter{Conclusion}\label{ch:conclusion}

This thesis introduces a reader to the problem of automated transcription of a multi-instrument sound recording to
sheet music. It discusses state-of-the-art solutions to the tasks of a source instruments separation, pitch
detection, key, and time signatures estimation, etc.

The goal has been reached at a sufficient level. The analysis and design of the implementation allow for future
expansion and improvement of the quality of the processing pipeline.

\section{Possible improvements}\label{sec:possible-improvements}

Implementation does not take into consideration the \textit{dynamics} of the notes. Dynamic of a sound describes its
amplitude or loudness (\pp, \p, \mf, \f, \ff, etc.), its emotional intensity and change through time (crescendo,
decrescendo). It is often an inalienable part of the description of the generated sound, hence a part of the output
sheet music. The solution to this task could be simple rule based or more complex that utilize \ac{ML} or statistical
analysis.

Another optimization for pitch detection could be usage of \textit{Multiresolution \ac{FFT}}. This method is widely-used
in a field of \ac{MIR} tasks. The frequency components of a \ac{DFT} are equally spaced and have a constant resolution.
However, in polyphonic music a higher frequency resolution is needed in the low and mid frequencies where there is
a higher density of harmonics. On the other hand, frequency modulation gets stronger as the number of harmonic is
increased, requiring shorter windows for improved time resolution~\cite{cancela2009efficient}. Thus, a multi resolution
spectral representation is highly desired for the analysis of music signals and can be a great improvement for the pitch
detection module of this implementation.