\chapter{Introduction}\label{ch:introduction}

The problem of automated sound transcription of sheet music increases its popularity each year. The usage of a system
capable to perform such task is very wide-ranging. It can be utilized by musical students to learn new musical peaces
that they have only in digital format (in \ac{WAVE} or MP3) or to train their pitch detection ability. The systems that
are able to perform musical key detection (like \textit{Mixed in Key}\cite{mixed-in-keys}) are commercially successful
and are very popular among DJs and music producers who use them for mixing the music. And in general music information
retrieving can bring a deeper insight into musical structure, quality, genre, etc. that can be used for many important
real-world problems, including recommendation systems for streaming platforms.

The problem of music transcription might be complicated even for a human listener, may be simpler for a trained
musician, but it appears very complex for the automated system. Since 1970, a whole research community has formed
and has been improving the quality of automated transcription and reduced the constraints required for producing it.

But it is still not a resolved problem and there is no such system that is capable of automatically and accurately
converting a recording of the musical piece back into sheet music or a set of commands that would replicate a recording
on a music synthesizer.

This thesis proposes the implementation and its architecture that attempts to solve the required tasks to generate
the correct transcription of the given musical piece. The architecture separates solutions for different problems into
modules which allows for simple extension of the system and testing of individual approaches for the given problems.
It also provides an estimated quality of each solution each evaluated on the appropriate datasets.

\section{Problem definition}\label{sec:problem-definition}

This thesis discusses the state-of-the-art solutions for the different problems that underlie a task of music
transcription, namely: source instruments separation, pitch and event detection, musical key, tempo and time signature
estimation that in combination determine necessary parts of the sheet music notation.

Short introduction to a sheet music notation is presented in \cref{ch:music_notation}.

\subsection{Pitch estimation}\label{subsec:pitch-estimation}
The difficulty of multi-pitch estimation lies in the fact that sound sources often overlap in time as well as in
frequency. Hence, the extracted information may be partly ambiguous. Also, when musical notes are played in harmonic
relations, the parts of higher notes may overlap completely with those of lower ones. Besides, spectral characteristics
of musical instrument sounds, that define change of the note volume in time and timbre, are diverse, which increases
the ambiguity in the estimation of amplitudes of sound sources. The resulting complexity leads not only to octave
ambiguity but also the ambiguity in the estimation of the number of sources.

\subsection{Tuning}\label{subsec:tuning}
Tuning, in general, is not a big problem for automated transcription systems. But it is important to note that recorded
musical instruments may not be in tune or all the notes are tuned higher or lower but keeping the correct ratios.
Estimation of a tuning, if used correctly, may improve the accuracy of the output score.

Another problem is that different tuning system can be used other than equal temperament, e.g.\ pythagorean tuning,
meantone temperament, well temperament, etc. But as most of the Western music uses equal temperament system, only this
system will be considered in the analysis.

\textit{Equal temperament} is a tuning system in which the interval between each pair of consequent notes has the same
ratio. Having chosen one base note and its frequency (commonly, \textit{A} with 440 Hz is used), frequencies of all
the other notes are defined to have $\sqrt[12]{2}$ ratio between adjacent notes, such that same note of a higher octave
would have twice the frequency, as there are 12 semitones in a standard 12 tone system.


\subsection{Key}\label{subsec:key}
Musical \textit{key} is a part of sheet music notation. It defines a set of used notes and chords in a piece of music
and their harmonic functions within the score. It simplifies the notation and understanding of the piece for a musician
who reads it. And even though the detection of a set of used notes is a fairly simple task, having notes detected within
a pitch estimation, classification of key is still a big research field as it is not as straight-forward task. The same
set of notes may indicate different keys depending on the context. For example, key of \textit{C major} has the same
set of notes as key of \textit{A minor} and distinguishing between them is often not a simple task even for a human
listener.

