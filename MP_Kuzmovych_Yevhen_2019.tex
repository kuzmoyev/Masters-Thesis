\documentclass[thesis=M,english,hidelinks]{FITthesis}[2012/10/20]

\usepackage[utf8]{inputenc} % LaTeX source encoded as UTF-8
\usepackage{graphicx} % graphics files inclusion
\usepackage{dirtree} % directory tree visualisation
\usepackage[printonlyused]{acronym} % acronims
\usepackage{float} % floats

\usepackage{amsmath} % for argmax, argmin
\DeclareMathOperator*{\argmax}{arg\,max}
\DeclareMathOperator*{\argmin}{arg\,min}
\usepackage{amssymb}

\usepackage{cleveref} % to use \cref wich includes words section or figure in reference
\usepackage{pdfpages}

\usepackage{wasysym} % in-text notes
\usepackage{harmony} % in-text notes
\usepackage{musixtex} % in-text notes

%\usepackage{indentfirst} % indent in first paragraph
\raggedbottom
%\usepackage[all]{hypcap} % for going to the top of an image when a figure reference is clicked
\usepackage[numbers]{natbib}
\usepackage{listings} % for the source code
\lstset{language=Python, frame=tb, basicstyle=\small} %or \small or \footnotesize etc.}


\graphicspath{{img/}} % images folder

\department{Department of Applied Mathematics}
\title{Multi-instrument music transcription}
\authorGN{Yevhen} %author's given name/names
\authorFN{Kuzmovych} %author's surname
\author{Yevhen Kuzmovych} %author's name without academic degrees
\authorWithDegrees{Bc. Yevhen Kuzmovych} %author's name with academic degrees
\supervisor{Ing. Marek Šmíd, Ph.D.}
\acknowledgements{THANKS (remove entirely in case you do not with to thank anyone)}
\abstractEN{Summarize the contents and contribution of your work in a few sentences in English language.}
\abstractCS{V n{\v e}kolika v{\v e}t{\' a}ch shr{\v n}te obsah a p{\v r}{\' i}nos t{\' e}to pr{\' a}ce v {\v c}esk{\' e}m jazyce.}
\placeForDeclarationOfAuthenticity{Prague}
\keywordsCS{Replace with comma-separated list of keywords in Czech.}
\keywordsEN{Replace with comma-separated list of keywords in English.}
\declarationOfAuthenticityOption{1} %select as appropriate, according to the desired license (integer 1-6)
% \website{http://site.example/thesis} %optional thesis URL


\begin{document}

%%%%%%%%%%%%%%%%%%%%%%%%%%%%% Custom commands %%%%%%%%%%%%%%%%%%%%%%%%%%%%%

\newcommand{\img}[3][1]{
\centerline{\includegraphics[scale=#1]{#2.#3}}
}

\newcommand{\textwideimg}[3][1]{
\centerline{\includegraphics[width=#1\textwidth]{#2.#3}}
}

% Usage:
%    \image[size]{diagram and lable name}{extention}{caption}
%    \image[1.3]{component_diagram}{pdf}{Component diagram}
\newcommand{\image}[4][1]{
\begin{figure}[H]
	\textwideimg[#1]{#2}{#3}
	\caption{#4}
	\label{fig:#2}
\end{figure}
}

\newcommand\figcenter[1]{
\begin{figure}[H]
	\centering
	#1
\end{figure}
}


\newcommand\inlinemusic[1]{\begin{music}#1\end{music}}

%%%%%%%%%%%%%%%%%%%%%%%%%%%%%%%%%%%%%%%%%%%%%%%%%%%%%%%%%%%%%%%%%%%%%%%%%%%%




\setsecnumdepth{part}
\chapter{Introduction}\label{ch:introduction}


\section{Problem definition}\label{sec:problem-definition}

\section{Tuning}\label{sec:tuning}


\setsecnumdepth{all}
\chapter{State-of-the-art}\label{ch:state-of-the-art}

This chapter discusses existing state-of-the-art solutions for music transcription.

Sound transcription into sheet music is a combination of several techniques that include but are not limited to source
instruments separation, pitch/note detection, event detection, etc.

Source separation and sound transcription to sheet music are fairly independent processes so their description and
approaches may come from different sources and different projects. Therefore, implementation will also be separated.

\section{Source separation}\label{sec:source-separation}

There were many successful attempts for music score source separation~\cite{spleeter2019,singing-voice-separation,singing-voice-separation-article}.
Performance of such projects are commonly measured according to \textit{\ac{SiSeC}}~\cite{stter20182018}
on the standard \textit{musdb18}~\cite{musdb18} and \textit{DSD100}~\cite{SiSEC16} datasets.

Latest and most successful project in this field is \textit{Spleeter}~\cite{spleeter2019}. It is a project of
Deezer\footnote{Deezer is a French online music streaming service (deezer.com).}. It takes similar approaches to previous solutions
by University of London and Spotify~\cite{singing-voice-separation}. Spleeter's pre-trained models will be used in the
module responsible for music source separation described in detail in the chapters~\ref{ch:analysis-and-design} and~\ref{ch:implementation}.

Following approaches are described in~\cite{spleeter2019,singing-voice-separation,singing-voice-separation-article}.

\subsection{Spleeter's approach}\label{subsec:music-source-separation:approach}
The pre-trained models are U-nets~\cite{singing-voice-separation} and follow similar specifications as in
\textit{Singing voice separation: a study on training data}~\cite{singing-voice-separation-article}. The U-net is an
encoder/decoder \ac{CNN} architecture with skip connections~\cite{spleeter2019}. Architecture used in this approach
showed a state-of-the-art results on DSD100 dataset~\cite{singing-voice-separation} and in the last \ac{SiSeC}~\cite{SiSEC16}.

\subsection{U-net architecture}\label{subsec:music-source-separation:u-net-architecture}
The U-Net shares the same architecture (shown in \cref{fig:u-net-architecture}) as a convolutional autoencoder with
extra skip-connections that bring back detailed information lost during the encoding stage to the decoding stage. It has
five strided\footnote{Transposed convolutions – also called \textit{fractionally strided convolutions} – work by
swapping the forward and backward passes of a convolution. One way to put it is to note that the kernel defines
a convolution, but whether it’s a direct convolution or a transposed convolution is determined by how the forward and
backward passes are computed~\cite{dumoulin2016guide}.} 2D convolution layers in the encoder and five strided 2D
deconvolution layers in the decoder.

The goal of the neural network architecture is to predict the vocal and instrumental components of its input indirectly:
the output of the final decoder layer is a soft mask for each source that is multiplied element-wise with the mixed
spectrogram to obtain the final estimate.

\image{u-net-architecture}{pdf}{Network Architecture~\cite{singing-voice-separation}}

\pagebreak

\subsection{Data and training}\label{subsec:music-source-separation:data-and-training}

Spleeter's training dataset is an internal Deezer's dataset and is not shared (for copyright reasons).

Another project with similar approach, as explained in the dedicated article~\cite{singing-voice-separation-article},
uses two datasets during training of the models: \textit{MUSDB}~\cite{musdb18} and \textit{Bean}(private dataset).

\textit{MUSDB} is the largest and most up-to-date public dataset for source separation~\cite{musdb18}. It contains 150
songs of western music genres primarily pop/rock, some hip-hop, rap and metal songs. And each song consists of four
audio tracks: drums, bass, vocal and other. Original mix (and input of the model) is produced by summing tracks of four
sources (expected outputs) together.


\section{Multi-pitch Detection}\label{sec:multi-pitch-detection}
There were several projects utilizing different approaches to a problem of \ac{AMT}. Following section discuss these
approaches and projects that used them.

The most important part of transcription of sound into sheet music is pitch (and subsequently note) detection. The core
problem of polyphonic music transcription is multi-pitch detection.

In his Ph.D.\ research \textit{Multiple fundamental frequency estimation of polyphonic recordings}~\cite{fundamental-frequency-estimation},
Chunghsin Yeh classifies multi-pitch detection systems according to their estimation type into two categories: joint and
iterative. The \textit{iterative} approach extracts the most eminent frequency per each iteration, until no other pitch
can be estimated and extracted. Commonly, iterative estimators generate errors on each iteration but are much cheaper in
terms of computation costs.

On the other side, the \textit{joint} estimation models evaluate combinations of pitches at once, which leads to
increase in accuracy but also in computation costs. Most of the latest approaches and state-of-the-art solutions fall
into the joint category. Solution in this thesis also follows this category and will be discussed in detail in \cref{ch:analysis-and-design}.

\subsection{Feature-based multi-pitch detection}\label{subsec:feature-based-multi-pitch-detection}
Most multi-pitch estimation and note tracking approaches exploit methods that come from signal processing. There is no
specific model (\ac{ML} or other), and notes are detected using audio features that come from the input time-frequency
representation (spectrogram) either in an iterative or joint way. Usually, multi-pitch estimation uses a \textit{pitch
candidate set score function} or a \textit{pitch salience function}.

A \textit{salience function} is a function that provides an estimation of the predominance of different frequencies in
an audio signal at every time frame~\cite{pitch-salience-function}. A \textit{pitch candidate set score function} is
a function designed to evaluate the plausibility of the combination of the hypothetical sources~\cite{fundamental-frequency-estimation}.

These feature-based techniques have produced the best results in the \ac{MIREX}~\cite{mirex} multi-pitch and note
tracking evaluations. The work by Chunghsin Yeh~\cite{fundamental-frequency-estimation} was the best performing method in
the \ac{MIREX} multi-pitch and note tracking tasks. Yeh proposed a joint pitch estimation algorithm based on a pitch
candidate set score function. Having a set of pitch candidates, the overlapping partials are detected and smoothed
according to the spectral smoothness principle, which states that the spectral envelope\footnote{\textit{Spectral
envelope} of the sound determines the particular vowel sound produced, and is, in general, one of the important acoustic
features that determine its perceived timbre~\cite{kumar2007hierarchical}.} of a musical sound tends to be slowly
changing as a function of frequency. The score function for the pitch candidate set consists of four features:
harmonicity, mean bandwidth, spectral centroid, and ``synchronicity'' (synchrony). A polyphony inference mechanism based
on the score function increase selects the optimal pitch candidate set~\cite{fundamental-frequency-estimation}.

In the following year, the best performing method for the \ac{MIREX} multi-pitch estimation and note tracking tasks,
Karin Dressler described in her work \textit{Multiple fundamental frequency extraction for MIREX}~\cite{dressler2012multiple}.
A multiresolution \ac{FFT} (see \cref{subsec:multiresolution-fft} on multiresolution \ac{FFT}) analysis was used as
an input time/frequency representation, where the magnitude for each spectral bin was multiplied with the bin’s
instantaneous frequency. Pitch estimation is made by identifying spectral peaks and performing pair-wise analysis on
them, resulting on ranked peaks according to harmonicity, smoothness, the appearance of intermediate peaks, and harmonic
number. Finally, the system tracks tones over time using an adaptive magnitude and a harmonic magnitude threshold.

Other notable feature-based \ac{AMT} solution was introduced in the work by Pertusa and Inesta \textit{Multiple
fundamental frequency estimation using Gaussian smoothness and short context}~\cite{pertusa2008multiple}. They proposed
a computationally inexpensive method for multi-pitch detection which computes a pitch salience function and evaluates
combinations of pitch candidates using a measure of distance between a \ac{HPS} and a smoothed \ac{HPS}. Another
approach for feature-based \ac{AMT} was proposed in \textit{Hybrid genetic algorithm based on gene fragment competition
for polyphonic music transcription}~\cite{reis2008hybrid}, which uses genetic algorithms for estimating a transcription
by mutating the solution until it matches a similarity criterion between the original signal and the synthesized
transcribed signal.

More recently, Peter Grosche et al. proposed~\cite{grosche2012automatic} an \ac{AMT} method based on a mid-level
representation derived from a multiresolution \ac{FFT} combined with an instantaneous frequency estimation. His system
also combines event (specifically start of the note) detection and tuning estimation for computing predictions. Finally,
Juhan Nam et al. proposed~\cite{nam2011classification} a classification-based approach for piano transcription using
features learned from deep belief networks~\cite{humphrey2013feature} for computing a mid-level time-pitch representation.

\subsection{Statistical model-based multi-pitch detection}\label{subsec:statistical-model-based-multi-pitch-detection}
Many approaches in the literature formulate the multi-pitch estimation problem within a statistical framework. As
Valentin Emiya et al. explains in their article \textit{Multipitch Estimation of Piano Sounds Using a New Probabilistic
Spectral Smoothness Principle}~\cite{proba-spectral-smoothness}: given an observed frame $\pmb{x}$ and a set $\pmb{C}$ of
all possible fundamental frequency combinations, the frame-based multi-pitch estimation problem can then be viewed as
a \ac{MAP} estimation problem:

\[ \hat{C}_{MAP} = \argmax_{C \in \pmb{C}} P(C|\pmb{x}) = \argmax_{C \in \pmb{C}} \frac{P(\pmb{x}|C)P(C)}{P(\pmb{x})} \]

where $C = \{F_0^1, \dots, F_0^N\}$ is a set of possible frequencies (considering tuning of an instrument), $\pmb{C}$ is
the set of all possible $F_0$ combinations, and $\pmb{x}$ is the observed audio signal within a single analysis frame.

An example of \ac{MAP} estimation-based transcription is the \textit{PreFEst} system introduced by Masataka Goto in his
article \textit{A real-time music-scene-description system: predominant-F0 estimation for detecting melody and bass
lines in real-world audio signals}~\cite{predominant-f0-estimation}, where each harmonic is modelled by a Gaussian
centered at its position on the log-frequency axis. \ac{EM} algorithm is used to estimate \ac{MAP} value.
An extension of this method was proposed by Kameoka et al. in \textit{A Multipitch Analyzer Based on Harmonic Temporal
Structured Clustering}~\cite{harmonic-temporal-structured-clustering}, which jointly estimates multiple possible
frequencies, moments of start and end of the note, and dynamics. Partials are modelled using Gaussians placed at
the positions of partials in the log-frequency domain and the synchronous evolution of partials belonging to the same
source is modelled by Gaussian mixtures.

More recently, Peeling and Godsill, in their article \textit{Multiple pitch estimation using non-homogeneous Poisson
processes}~\cite{peeling2011multiple}, also proposed a likelihood function for multiple-pitch estimation where for a
given time frame, the occurrence of peaks in the frequency domain is assumed to follow an inhomogeneous Poisson process.
Also, Koretz and Tabrikian in \textit{Maximum a posteriori probability multiple-pitch tracking using the harmonic
model}~\cite{koretz2011maximum}, proposed an iterative method for multi-pitch estimation, which combines \ac{MAP} and
\ac{ML} criteria. The predominant source is expressed using a harmonic model while the remaining harmonic signals are
modelled as Gaussian interference sources~\cite{koretz2011maximum}.

\pagebreak

\subsection{Spectrogram factorisation-based multi-pitch detection}\label{subsec:spectrogram-factorisation-based-multi-pitch-detection}
The majority of recent multi-pitch detection papers utilise and expand spectrogram factorisation techniques. \ac{NMF} is
a technique first introduced in their paper by Paris Smaragdis et al.~\cite{smaragdis2003non} as a tool for music
transcription. In its simplest form, the \ac{NMF} model decomposes an input spectrogram
$\pmb{X} \in \mathbb{R}^{K \times N}_+$ with $K$ frequency bins and $N$ frames: \[ \pmb{X} \approx \pmb{WH} \]
where $R \ll K, N$; $\pmb{W} \in \mathbb{R}^{K \times R}_+$ contains the spectral bases for each of the $R$ pitch
components; and $\pmb{H} \in \mathbb{R}^{R \times N}_+$ is the pitch activity matrix across time.

In his paper \textit{Realtime multiple pitch observation using sparse non-negative constraints}, Cont Arshia applies
\ac{NMF} to \ac{AMT} problem. Sparseness constraints were added into the \ac{NMF} update rules, in order to find
meaningful transcriptions using a minimum number of non-zero elements in $\pmb{H}$. Emmanuel Vincent et al. in their
article \textit{Adaptive harmonic spectral decomposition for multiple pitch estimation}~\cite{vincent2009adaptive}
incorporated harmonicity constraints in the \ac{NMF} model, resulting in two algorithms: harmonic and inharmonic
\ac{NMF}. The inharmonic version of the algorithm is also able to support deviations from perfect harmonicity and
standard equal temperament tuning. Also, Nancy Bertin et al. in their article~\cite{bertin2010enforcing} proposed
a Bayesian framework for \ac{NMF}, which considers each pitch as a model of Gaussian components in harmonic positions.

More recently, Ochiai et al. in his paper \textit{Explicit beat structure modeling for non-negative matrix
factorization-based multipitch analysis}~\cite{ochiai2012explicit} proposed an algorithm for multi-pitch detection and
beat structure analysis. The \ac{NMF} objective function is constrained using information from the rhythmic structure of
the recording. It helped to improve transcription accuracy in highly repetitive recordings.

This thesis approaches \ac{AMT} problem in similar fashion to spectrogram factorisation and feature-based methods.
Detailed description of used methods is in \cref{ch:analysis-and-design}.

\section{Note Tracking}\label{sec:note-tracking}

Typically \ac{AMT} algorithms compute a time-pitch representation which needs to be further processed in order to detect
note events with a discrete pitch value, a time of start and end of the note. This process is called \textit{note
tracking}. Most spectrogram factorisation-based methods estimate the binary piano-roll representation from the pitch
activation matrix using simple thresholding (i.e.\ in \textit{Explicit beat structure modeling for non-negative matrix
factorization-based multipitch analysis}~\cite{grindlay2011transcribing} by Graham Grindlay and in \textit{Adaptive
harmonic spectral decomposition for multiple pitch estimation}~\cite{vincent2009adaptive} by Emmanuel Vincent). This
approach will be used in the implementation of the thesis. Also, one simple and fast optimisation for note tracking is
minimum duration pruning, which is applied after thresholding (idea comes from paper by Arnaud Dessein et al.
\textit{Real-time polyphonic music transcription with non-negative matrix factorization and beta-divergence}~\cite{dessein2010real}).
Primary idea is that output notes that have a duration smaller than a predefined threshold are removed from the final
score. Similar method was also used by Juan Pablo Bello et al. in their paper \textit{Automatic piano transcription
using frequency and time-domain information}~\cite{bello2006automatic}, where more complex rules for note tracking were
used, addressing cases such as where a small gap exists between two note events. This method will also be applied as it
is easy to implement.

For threshold based method, there are several issues that may appear for different instruments, primarily related to how
notes are used and written for them in practice. For instance, for percussion instruments, note decay is exponential and
physical duration of the note is irrelevant as it is not controlled(for most percussion instruments) by a musician. This
way notes may appear short and require pauses(rests) after them, even though the rests would not be written in sheet
music. Such problems may be solved with other rule based solutions specific to each instrument that requires them or
more complex approaches.

Even though a simple threshold-based solution was used for the note tracking task, following paragraph discusses some
more complex and more accurate solutions, though without detailed description of the approaches.

\acp{HMM} are frequently used at a stage of postprocessing of note tracking. In his work \textit{A discriminative model
for polyphonic piano transcription}~\cite{poliner2006discriminative}, Graham Poliner proposes a note tracking method that
utilizes pitch-wise \acp{HMM}, where each \ac{HMM} has two states, indicating note activity and inactivity. \ac{HMM}
parameters (state transitions and priors) were learned directly from a ground-truth training set, while the observation
probability is given by the posteriogram output for a specific pitch.

In \textit{Polyphonic music transcription using note event modeling}~\cite{ryynanen2005polyphonic} by Matti P Ryynanen
and Anssi Klapuri, a feature-based multi-pitch detection system was combined with a musicological model for estimating
musical key and note transition probabilities. Note events are described using 3-state \acp{HMM}, which model
the envelope (attack, sustain, and noise/silence states) of each sound. In addition, context-dependent \acp{HMM} were
employed in \textit{Automatic transcription of recorded music}~\cite{grosche2012automatic} for determining note events by
combining the output of a multi-pitch detection system with a note-start detection system.

Finally, \acp{DBN} were proposed by Shigeki Sagayama et al. in their paper~\cite{raczynski2009note} for note tracking.
They used the pitch activation of a \ac{NMF}-based multi-pitch detection algorithm as input. The \ac{DBN} has a note
layer in the lowest level, followed by a note combination layer. Model parameters were learned using MIDI files from
F. Chopin piano pieces.

\section{Tuning, time signature, key, and tempo estimation}\label{sec:other-amt-subtasks}
There are several other subtasks of \ac{AMT} systems that have to be resolved to be able to generate correct
transcription in a form of sheet music. Also, such estimates, if properly incorporated to the system, may improve
estimates of detected pitches and their durations, events. Tuning, time signature, key, and tempo estimation are such
tasks.

\subsection{Key and chord detection}\label{subsec:key-and-chord-detection}
Most Western music has a harmonic organisation around one key. The key is generally unchanged over whole, or at least
sections of musical pieces. At one point in time, the harmony may be described by chord, which is a combinations of
simultaneous or sequential notes which are perceived to belong together (and in general sound nice together, even though
any combination of notes has its chord). Algorithms for key (and similarly for chord) detection use template matching or
\acp{HMM}. For key detection, this thesis uses the simple approach defined in \cref{sec:key-classification}.

\subsection{Tempo and time signature estimation}\label{subsec:tempo-and-time-signature-estimation}
The \textit{beats} are regularly spaced in time pulses. They are the primary unit that defines tempo and rhythm of most
Western music. A number of beats per unit of time (commonly per minute in sheet music) defines a \textit{tempo}. A number
of beats per uniform repetitive sections in score (bars) defines a \textit{time signature}. In order to interpret an audio
recording in terms of such a structure (which is necessary in order to produce Western music notation), the first step
is to determine the rate of the most salient pulse, which is the tempo.

Algorithms used for tempo induction include autocorrelation, Fourier transforms, and periodicity transform, which are
applied to audio features such as a note-start detection function (as Fabien Gouyon and Simon Dixon describe in their
article \textit{A review of automatic rhythm description systems}~\cite{gouyon2005review}). The next step involves
estimating the timing of the beats constituting the main pulse, a task known as beat tracking. Again, numerous
approaches have been proposed, such as rule-based methods (as in \textit{Computational models of beat induction:
The rule-based approach}~\cite{desain1999computational} by Peter Desain and Henkjan Honing), adaptive oscillators (as
in \textit{Resonance and the perception of musical meter}~\cite{large1994resonance} by Edward W Large and John F Kolen),
agent-based or multiple hypothesis trackers (as in \textit{Automatic extraction of tempo and beat from expressive
performances}~\cite{dixon2001automatic} by Simon Dixon), and other.

B\"{o}ck et. al. proposed a novel tempo estimation algorithm based on a recurrent neural network that learn an intermediate
beat-level representation of the audio signal which is then feed to a bank of resonating comb filters to estimate
the dominant tempo~\cite{madmom}. This algorithm got the best score in ISMIR 2015 Audio Tempo Estimation task.
The implementation by the authors is included in the opensource \textit{Madmom} audio signal processing library which
will be used in the implementation.

The final step for rhythmic analysis consists of estimating the time signature, which indicates how beats are grouped
and subdivided at respectively higher and lower metrical levels, and assigning each note-start and offset time to
a position in this structure~\cite{cemgil2011monte}.



\chapter{Analysis and design}\label{ch:analysis-and-design}

This chapter defines architecture of the chosen solution. It provides details of used approaches for music sources
separation and models used in it, pitch extraction and signal analysis, event detection, etc.

\section{Architecture}\label{sec:architecture}

The implementation of the system is separated into logical parts responsible for sound data streaming, music source
separation, pitch and events detection, transcription and score generation. Following diagram shows architecture of
the solution. Arrows represent data flow. Dotted arrows represent flow that is optional. If given parameters (like
tuning, tempo, time signature and key) are specified by user, they are not being estimated. Each rectangular block
represents logical module in implementation.

Detailed description of each component is in the dedicated section following the diagram on \cref{fig:architecture}.

\image[0.74]{architecture}{pdf}{Architecture of the implementation.}

\section{Audio streaming}\label{sec:audio-streaming}

This implementation directly works only with \ac{WAVE} (.wav/.wave). Any other format is converted to WAVE first,
then processed.

\subsection{WAVE format}\label{subsec:wave-format}
\textbf{WAVE} is an audio file format standard, developed by Microsoft and IBM, for storing an audio bitstream on PCs.
What's important for this thesis and implementation is that it stores data in chunks in \ac{LPCM} format. This format
allows to perform \ac{DFT} used in pitch extraction.

\subsection{Sampling rate}\label{subsec:sampling-rate}
\ac{LPCM} mentioned above stores sampled amplitude of recorded audio at specific sampling rate (frequency, measured in
Hz).

The most common sampling rate is 44.1 kHz, or 44100 samples per second. This is the standard for most consumer audio,
used for formats like CDs~\cite{digital-audio-basics}.

The sampling rate determines the range of frequencies captured in digital audio. The lowest frequency a person can hear
is 20 Hz. The highest frequency humans can hear are in the range of 20.000 Hz, but only young people can hear such high
tones~\cite{roots-of-modern-technology}. According to \textit{Nyquist Theorem}, a signal which has a Fourier transform
having only frequencies upto a certain maximum $f_m$, we can obtain the analog signal $f(t)$ from the sampled signal
$f'(t)$ by passing the sampled signal $f'(t)$ through a low pass filter provided that the sampling frequency $f_s$ is
more than twice the maximum frequency $f_m$ present in the signal i.e.\ , $f_s > 2f_m$~\cite{signals-and-systems}. Hence,
having 44100 Hz sampling rate, we can reproduce and analyse frequencies up to 22050 Hz (assuming an ideal low pass
filter). Otherwise, if recorder has a sampling rate lower than $2\times$ the highest frequency (which was not cut off
by low pass filter) it causes the effect called \textit{aliasing}, which introduces unexpected sounds in the recording
that were not present in the original sound. If the sampling frequency is too low the frequency spectrum overlaps, and
becomes corrupted~\cite{signals-and-systems}.

The implementation is able to process input sound with any sampling rate, though lower sampling limits processed
frequencies range to lower pitches.

\pagebreak

\section{Music source separation}\label{sec:music-source-separation}

First step of the sound processing is separation of the sound into source instruments (i.e.\ voice, guitar, piano,
etc.)

As was mentioned in the previous chapter, this implementation uses \textit{Spleeter} for separation of source
instruments. \textit{Spleeter} is a fast and state-of-the art music source separation tool with pre-trained
models~\cite{spleeter2019}. Its implementation contains three pre-trained models:

\begin{itemize}
	\item vocals/accompaniment separation,
	\item 4 stems separation as in \ac{SiSeC}~\cite{stter20182018} (vocals, bass, drums and other),
	\item 5 stems separation with an extra piano stem (vocals, bass, drums, piano and other). It is, to the author's
	knowledge, the first released model to perform such a separation.
\end{itemize}

Estimations for all the models is performed in a frequency domain of the sound. Meaning that sound data from time domain
is converted to frequency domain using \ac{FFT}, passed to the models described in section~\ref{subsec:music-source-separation:u-net-architecture}
about U-net architecture. Output of the model is separated tracks for each instrument and voice. To get sound of each
instrument and voice in time domain (as it would be represented in \ac{WAVE}), we would need to pass it through
\ac{IDFT}. But it is not necessary, as all the subsequent processing will be performed on the sound in frequency domain.

More about \ac{FFT} is in the following section~\ref{sec:pitch-extraction} about pitch extraction.

\section{Pitch extraction}\label{sec:pitch-extraction}
As was mentioned in \cref{sec:multi-pitch-detection} there are many approaches to pitch (and specifically
to multi-pitch) detection. The one that is presented in this theses utilizes a combination of ideas defined in works
of Matti P Ryynanen et al.~\cite{ryynanen2005polyphonic}, Arnaud Dessein et al.~\cite{dessein2010real} and Paris Smaragdis
et al.~\cite{smaragdis2003non}. Solution is joint, thus estimates played notes at a given moment all at once (opposed to
iterative approaches). It attempts to detect frequencies similarly to matrix factorization techniques through analysis
of sound spectrogram. But instead of matrix factorization (\ac{NMF}), this work attempts to detect notes' events,
specifically their envelopes (more on the sound envelope in \cref{subsec:sound-envelope}), using \ac{ML} models.
Specification of the used data and training of the models is in the \cref{subsec:data-and-model-training}.

\pagebreak

\subsection{Sound envelope}\label{subsec:sound-envelope}
\textit{Sound envelope} is a variation of the sound volume in time~\cite{dregvaite2015information}. Sound envelope
consists of 4 stages: \ac{ADSR}:

\image{sound-envelope}{pdf}{Sound envelope.}

\cref{fig:sound-envelope} shows a theoretical simple \ac{ADSR} model of sound envelope. But different instruments
produce different envelopes depending on a nature of sound extraction:

\image{instrument-envelopes}{png}{Sound envelopes of piano and violin~\cite{sound-envelope}.}

As seen on the \cref{fig:instrument-envelopes}, piano and any other instrument that produces sound by hitting, tapping
or pinching of a string (like guitar, harp, bandura, balalaika, etc.), will produce similar envelope with defined
attack (the moment of piano key pressing; pinching or hitting a string on guitar, etc.), decay and sustain (when piano
key remains pressed or piano sustain pedal is used, etc.) and release (when piano key and sustain pedal are released,
guitar strings are muted, etc.).


\section{Event detection}\label{sec:event-detection}

Note and subsequently its pitch and start are estimated by detecting its envelope. Having a sample of sound (change of
volume of each pitch as determined from \ac{FFT}) of duration $k$ seconds specified by implementation, predictive model
attempts to estimate whether note has been played at a given point in time by detecting its envelope that should look
similar for each note of the given instrument. That means that there will be a model for each predefined instrument
trained on its samples (more in \cref{subsec:data-and-model-training}).

\subsection{Data and model training}\label{subsec:data-and-model-training}
Dataset for training of the above-mentioned models was generated from the \textit{NSynth dataset}~\cite{nsynth2017}.
\textit{NSynth} is an audio dataset containing 305,979 musical notes, each with a unique pitch, timbre, and envelope.
For 1,006 instruments from commercial sample libraries, there are generated four second, monophonic 16kHz audio
snippets, referred to as notes, by ranging over every pitch of a standard MIDI piano (21-108) as well as five different
velocities (25, 50, 75, 100, 127)~\cite{nsynth2017}.

NSynth contains samples for 11 different instruments: bass, brass, flute, guitar, keyboard, mallet, organ, reed,
string, synth\_lead, vocal. They are all stored in \ac{WAVE} format and have needed metadata in JSON format alongside
with them. Metadata for each sample includes instrument, note, pitch and velocity in \ac{MIDI} format (in the range
[0, 127]), and sampling rate.

Spleeter, used for source separation, is able to separate sound only into 5 source instruments. Hence only those samples
from NSynth will be used to generate models.

The preprocessing of the training dataset is completely the same as preprocessing of the sound samples during
transcription. The whole data flow is shown on \cref{fig:pitch-detection}.

\image[1.15]{pitch-detection}{pdf}{Data flow for pitch detection.}

As shown on \cref{fig:pitch-detection}, input stream (blue) is a stream of data read from input file or microphone (or
any other input). It is read by chunks of size $c$ determined by implementation. Each window of $k$ chunks is passed
through \ac{FFT} to transform data to a frequency domain. Taking several chunks of data to pass it through \ac{FFT}
increases its accuracy, peaks of played pitches become more prominent and output becomes more robust to noise and phase
shifts. Overlapping of $k$-sized windows allows producing more data-points per second and subsequently features for
predictive model. Having $T$ chunks and window of size $k$, produced spectrogram is of size $F,T-k$ where $F$ is
a number of detected frequencies.

After transforming input to time-frequency spectrogram (red), frequencies are translated to musical notes (green).
Assuming equal temperament tuning and \textit{A} with 440Hz frequency (actual tuning will be estimated later in
the analysis), frequencies are converted to the closest note. Frequencies converted to the same note are filtered
to leave only the highest volume value.

The output of previous step is passed to the model of a given instrument by window of size $m$. So $m$ is a number of
input features of the model. The model attempts to classify whether given window contains an envelope of a played sound
that starts from a given point in time. So for $\forall{i} \in [0,N], j \in [0,T-k-m]$, $p_{i,j}$ (pink) shows
prediction of the model for note $i$ at a time $j$.

Training data is generated from NSynth dataset in a similar fashion. Positive labels are set for the pitch being played
in a sample, negative for all the others. Also negative examples are generated from the same sample for played pitch but
with a time shift, starting the example from or ending it somewhere in the middle of the actual sound envelope.


\section{Tuning classification}\label{sec:tunning-classification}
Tuning of the instrument is not a part of score transcribed into sheet music and most of the Western music follows
the same tuning system. Specifically, equal temperament system with \textit{A} tuned to 440Hz. But tuning estimation is
an essential part for correct reproduction of the sound.

There are two primary parts to tuning estimation: detection of frequency of base note, commonly \textit{A}:
\figcenter{
\lilypondfile{ly/analysis_and_design/base_a.ly}
\caption{Base \textit{A} commonly tuned to 440Hz.}
}
and tuning system, like equal temperament, pythagorean, meantone, etc. While tuning system in the Western music very
rarely diverges from equal temperament, frequency of a base note can often be chosen to be different from 440Hz. That
also often might happen for instruments that are often being retuned like guitars as tuning ``by ear'' by person that
does not have a perfect pitch\footnote{\textit{Perfect pitch} or \textit{absolute pitch} is the ability to identify
a note by hearing it. The ability is considered remarkably rare, estimated to be less than one in 10,000
individuals~\cite{perfect-pitch}.} is defined by tuning of the string relative to which all other strings are tuned.

Having found one played frequency $F$ in sound (or several frequencies for better precision), it is matched to
the closest note $N$ in 440Hz-A-tuning, then real frequency of \textit{A} note $f(A)$ is calculated as:

\[ f(A) = F * 2^\frac{t(N)-t(A)}{12} \]

where $t(N)$ is an index number of the note(semitone) $N$ (for example in \ac{MIDI} representation). Now, all the other
notes can be calculated in the same way.

\section{Tempo estimation}\label{sec:tempo-estimation}
As was mentioned in \cref{subsec:tempo-and-time-signature-estimation}, \textit{Madmom} library will be used for
the task of tempo estimation. Authors use a recurrent neural network to learn an intermediate beat-level representation
of the audio signal. The output of the neural network is a beat activation function, which represents the probability of
a frame being a beat position. And instead of processing the beat activation function to extract the positions of
the beats, authors use it directly as a one-dimensional input to the bank of resonating comb filters. \text{Comb
filtering} can be defined as ``the frequency response caused by combining a sound with its delayed duplicate.
The frequency response displays a series of peaks and dips caused by phase interference. The peaks and dips look like
the teeth in a comb, with very narrow, deep notches where signals are attenuated.''~\cite{comb-filter}. Using comb
filters with different lags (delays) implementation of madmom detects at which lag the beat of the sound resonates
the most. Given lag would then define a tempo.

The range of possible tempo values (\ac{BPM}) $t$ is limited to $1 \le t \le 128$ and to only whole numbers. This is
decided so that the range can include loops that last from only 1 beat to 128 beats, which would correspond to a maximum
of 32 bars in $4\atop4$ meter.

\section{Time signature estimation}\label{sec:time-signature-estimation}
Problem of time signature or \textit{meter} estimation is similar to tempo detection in a sense that the basic idea of
it is finding its repetitiveness, recurrence - beat for tempo and content of a sheet music bar for time signature. That
is why solutions for these problems often overlap.

Another approach for tempo as well as for time signature estimation is autocorrelation modeling. Autocorrelation
modeling is used to determine the length of the bar - number of beats per each meter. Technically, time signature
definition can contain any numbers for number of beats per measure (top number) and the note value that receives one
beat (bottom number). But most of music peaces of western music use powers of 2 as a note value (otherwise it is called
irrational measure) and rarely higher than 8 (\Acht). As for number of beats per bar, a 1 or values higher than 12 are
considered to be the unusual time signatures. So the implementation limits estimation to these ranges.

Knowing the tempo - a number of fourth notes (\Vier) per minute, taking an average volume of the notes (0 if no notes
are there) in all position in time of sixteenth notes(\Sech) produces the time series on which implementation models
autocorrelation. Assuming that rhythmical structure of the bar and position of strong and weak\footnote{Commonly, some
notes per bar are \textit{strong}(louder) and some are \textit{weak}(quieter). This determines accents in measure. For
example in $4\atop4$ time, first beat is often the loudest(strong), third is also strong, but not as strong as
the first, and second and fourth are weak.} beats is continues through the whole peace or its significant part, the lag
of modeled autocorrelation will determine the number of sixteenth notes per bar.

\ac{ARIMA} model is used to determine the lag. \ac{ARIMA} is a class of models that ``explains'' a given time series
based on its own past values, that is, its own lags (AR part) and the lagged forecast errors (MA part), so that equation
can be used to forecast future values. Specifically its simpler version AR will be used. AR is defined as follows:
\[ Y_t = \alpha + \beta_1*Y_{t-1} + \beta_2*Y_{t-2} + \dots + \beta_n*Y_{t-n} + \epsilon_1 \]
where $Y_t$ is value measured in time $t$, $\alpha$ is the intercept term, $\beta_k$ is coefficient of the first lag,
and $\epsilon$ is a noise. All of $\beta$s and $\alpha$ are estimated by the model. The higher the absolute value
$\beta_k$ - the higher the correlation between the signal and its copy delayed on lag $k$. Obviously the highest
correlation of a signal is with its 0 lag. But as was mentioned above, the choice is limited to range $2 \le k \le 12$
with 8 as a shortest note value. So the coefficient $\beta$ are estimated from 8th up to 32nd lag to determine time
signatures from $2\atop4$ (which in time is equivalent to $8\atop16$) up to $12\atop8$ (which in time is equivalent to
$24\atop16$) and up to $8\atop4$ or $4\atop2$ (which in time are equivalent to $32\atop16$).

Technically any score for which appropriate $k$ was found, can be written within $k\atop16$ measure. But it is better to
identify the best logical value for the number of beats per measure for simplification of reading of the score and
convert the time signature to either $k/2\atop8$, $k/4\atop4$, or $k/8\atop2$.

Having found a value of $k$, converting is performed according to a \cref{fig:time-signatures-table} as indicated by
green cells for even values of $k$. Odd values of $k$ are not expected as they are very unusual in measures of
$k\atop16$. If they are odd time signature is left as is. Important to note that measures indicated by green cell are
more common than their counterparts within a row but time signatures like $6\atop8$ and $2\atop2$ are also widely used
in music. But it is hard to objectively identify which of the measures $6\atop8$ or $3\atop4$, $2\atop2$ or $4\atop4$
should be used.

\begin{table}[]
	\begin{center}
		\def\arraystretch{1.3}
		\begin{tabular}{|c||c|c|c|c|}
			\hline
			k & 16 (\Sech)   & 8 (\Acht)   & 4 (\Vier)   & 2 (\Halb)  \\ \hline
			8 & $8\atop16$   & $4\atop8$    & \cellcolor[HTML]{9AFF99}$2\atop4$   & $1\atop2$  \\ \hline
			10 & $10\atop16$  & \cellcolor[HTML]{9AFF99}$5\atop8$    & &            \\ \hline
			12 & $12\atop16$  & $6\atop8$    & \cellcolor[HTML]{9AFF99}$3\atop4$   &            \\ \hline
			14 & $14\atop16$  & \cellcolor[HTML]{9AFF99}$7\atop8$    & &            \\ \hline
			16 & $16\atop16$  & $8\atop8$    & \cellcolor[HTML]{9AFF99}$4\atop4$   & $2\atop2$  \\ \hline
			18 & $18\atop16$  & \cellcolor[HTML]{9AFF99}$9\atop8$    & &            \\ \hline
			20 & $20\atop16$  & $10\atop8$   & \cellcolor[HTML]{9AFF99}$5\atop4$   &            \\ \hline
			22 & $22\atop16$  & \cellcolor[HTML]{9AFF99}$11\atop8$   & &            \\ \hline
			24 & $24\atop16$  & $12\atop8$   & \cellcolor[HTML]{9AFF99}$6\atop4$   & $3\atop2$  \\ \hline
			26 & $26\atop16$  & \cellcolor[HTML]{9AFF99}$13\atop8$   & &            \\ \hline
			28 & $28\atop16$  & $14\atop8$   & \cellcolor[HTML]{9AFF99}$7\atop4$   &            \\ \hline
			30 & $30\atop16$  & \cellcolor[HTML]{9AFF99}$15\atop8$   & &            \\ \hline
			32 & $32\atop16$  & $16\atop8$   & $8\atop4$   & \cellcolor[HTML]{9AFF99}$4\atop2$  \\ \hline
		\end{tabular}
		\caption{Time signature selection table.}
		\label{fig:time-signatures-table}
	\end{center}
\end{table}


\section{Key classification}\label{sec:key-classification}
Key signature is a part of sheet music notation that simplifies it by avoiding redundant repetitive accidentals (sharps
and flats) and by defining the set of used notes which in its turn defines the set of used chords, their progressions
and harmonic functions in a piece of music. Detailed description of key signature is in the \cref{subsec:key-signature}.

As was mentioned in \cref{subsec:key-and-chord-detection}, there are several approaches to key classification including
template matching or \acp{HMM}. The best paid solution for key detection is \textit{Mixed In Key}~\cite{mixed-in-keys}
having 95\% accuracy on their test dataset. The best free solution is \textit{KeyFinder}~\cite{key-finder} with accuracy
of 77\%. But it is implemented in C++ so is hard to incorporate into Python implementation used in this work.

A simple heuristical solution was used in the framework of this thesis. For each note that has a sharp counterpart
(\textit{C}, \textit{D}, \textit{F}, \textit{G}, and \textit{A}), if its semitone-higher (sharper) note appears more
often than its natural note in any octave, its sharp symbol is included into a key signature. For example:
\figcenter{
\lilypondfile{ly/analysis_and_design/key_signature_c.ly}
\caption{Notes in a key of \textit{C major}.}
}
will be translated into
\figcenter{
\lilypondfile{ly/analysis_and_design/key_signature_g.ly}
\caption{Notes in a key of \textit{G major} (note a $\sharp$ on the \textit{F} line).}
}

Notes above are one \textit{F} (natural) and two of \textit{F}$\sharp$ from three different octaves. As there are more
\textit{F}$\sharp$ notes, output transcription will be in a key of \textit{G major}, that consists of notes \textit{G},
\textit{A}, \textit{B}, \textit{C}, \textit{D}, \textit{E}, and \textit{F}$\sharp$.

It is important to note that \textit{E minor} key has the same set of notes, but to determine whether key is
\textit{G major} or \textit{E minor} is a much more complex task and requires an analysis of chord progressions and
their harmonic functions within the context of a given score.

\section{Post processing}\label{sec:post-processing}

Duration of the note is determine by its start (start of the sample passed to the model) and its end (moment, when
note's volume lowers under the predefined threshold). As was mentioned in \cref{ch:state-of-the-art}, implementation
also utilizes several simple postprocessing ideas:
\begin{itemize}
	\item if the duration of the note is too short, it is omitted,
	\item if the duration between end and next start of the same note is too short, it is combined into a single note,
	\item if the note played at the same time with another note but with the significant difference in volume,
	the quieter note is omitted.
\end{itemize}

Another goal of post processing module is the determination of the rests positions and their lengths. Rests fill in
the gaps in a bar where no note is played. They are required to position notes correctly on the staff and subsequently
in time for the musician.
As for notes, it is needed to determine time of the rests' start and their length (whole rest, half rest, quoter rest,
etc.). The process is fairly straightforward having the notes and their positions detected:

\figcenter{
\begin{algorithm}[H]
	\For{each bar}{
		\While{not the end of the bar}{
			\eIf{any note starts at a current position}{
				go to the end of the note\;
			}{
				mark start of the rest\;
				go to the next note or an end of the bar whatever comes first and mark it as an end of a rest\;
				determine the length of the rest from its start and end\;
			}
		}
	}
\end{algorithm}
\caption{Rests detection algorithm.}
}

\section{Score generation}\label{sec:score-generation}

Finally, having estimated time signature, key, positions and pitches of notes, rests, the output transcription is
generated.

\textit{LilyPond}~\cite{lilypond} is used for score generation. Lilypond is a computer program and file format for music
engraving. Notes in Lilypond are represented in pitch-duration format: pitch is specified with \textit{Helmholtz pitch
notation}\footnote{\textit{Helmholtz pitch notation} is a system for naming musical notes. For example, the note C is
shown in different octaves by using upper-case letters for low notes, and lower-case letters for high notes, and adding
sub-primes and primes in the following sequence: $C_{\prime\prime}$ $C_\prime$ $C$ $c$ $c'$ $c''$ $c'''$, where $c'$ is
the \textit{middle C}.}, and duration is specified with a \textit{numeral based system}\footnote{Duration of a note is
specified with numbers 1 (\Ganz), 2 (\Halb), 4 (\Vier), 8 (\Acht), etc. For a quarter note (\Vier) number can be
omitted.}.

The output of this module is a Lilypond file (in .ly format) and subsequently score in \ac{PDF}.

\chapter{Implementation}\label{ch:implementation}

This chapter provides details of the implementation, used tools, training and testing of the models.


\section{Used tools}\label{sec:used-tools}

\textit{Python} was used as a primary programming language for implementation of the project. Input streams are
processed by \textit{pyaudio} library. Data is stored in a \textit{numpy} array in 16 bit integers. \textit{Numpy} and
\textit{Pandas} were used for datasets processing. \textit{Scipy}'s implementation of \acp{DNN} (and other models that
were tested) was used for models of event detection module. Models were trained in \textit{jupyter notebook} included
in the implementation sources. Models are serialized into .pickle format using Python's \textit{joblib} module.
The \textit{statsmodels} library is used to generate \ac{ARIMA} model used in time signature estimation

As was discussed in \cref{ch:analysis-and-design}, implementations of \textit{Spleeter} and \textit{Madmom} were used
for source splitting and tempo detection modules respectively. Output score in lilypond format is generated using
\textit{abjad} that has Pythonic object-oriented interface for sheet music engraving which uses the lilypond compiler.



\section{Project structure}\label{sec:project-structure}
Project follows common approaches for Python project structures and implements a Python module as well as \ac{CLI} for
music transcription. It is well parametrized such that user can define a set and tuning of the instruments, tempo, time
signature and key. Otherwise these parameters are being estimated by dedicated modules according to design described in
a \cref{ch:analysis-and-design}.

The main endpoint that is used by \ac{CLI} and may be used by developers is \textit{mimt.music\_transcription} (mimt
stands for multi-instrument music transcription). It is responsible for the whole data pipeline that starts at input
stream reading.

\subsection{Input stream}\label{subsec:input-stream}
Input stream can have any source: file, microphone, or any other source which implements
\textit{audio\_reading.\_AbstractStream} interface. For microphone and file streaming there are already implemented
classes \textit{MicrophoneStream} and \textit{FileStream} respectively. Implementations are parametrized by size of read
chunks, but it is important to note that divergence from default value will affect the representation of the data which
in its turn would require retraining of the models used in event detection.


\subsection{Music source separation and tempo estimation}\label{subsec:music-source-separation-and-tempo-estimation}
Modules of the implementation utilize existing solutions for problems of source separation (\textit{Spleeter}) and tempo
estimation (\textit{Madmom}). There are a dedicated modules for both of the problems. The modules are just calling
needed functions from implementations of \textit{Spleeter} and \textit{Madmom}. Hence, modules are simple and were
included into a project as a dedicated Python modules only to be consistent within a structure and interfaces of
the implementation.

By default, this implementation uses five stems separation. Otherwise it can be defined as a parameter of the analysis.
If any of the output sheet music scores does not have any notes, it is ignored and does not have output sheet music.

\subsection{Pitch detection}\label{subsec:pitch-detection}
Following the source separation module, pitch extraction transforms signals of each instrument from time domain to
a frequency domain using \ac{FFT}. Numpy's \textit{fft} submodule is used to perform the transformation. Amplitudes
of frequencies generated by \ac{FFT} then converted to decibels with the following formula:
\[volume = 10 * \log_{10}amplitude\]
\textit{PitchExtractor} stores the frame rate of the input stream to correctly convert frequencies produced by \ac{FFT}
to real frequencies of a sound (Hz):
\[frequency = fft\_frequency * frame\_rate\]

Frequencies are then converted to the closest notes. Assuming equal temperament tuning system and note A tuned to
440 Hz, each frequency is converted to a note in a format used in \textit{abjad} modules, specifically
\textit{\{note\_name\}\{octave\}}, e.g. \textit{A2} is 440 Hz \textit{A} note, \textit{C2} is the \textit{middle C},
\textit{Cs3} is a \textit{C$\sharp$} - one octave and a semitone higher of the \textit{middle C}, etc. First,
the semitone number above or bellow \textit{440-Hz-A} is calculated:
\[n = \lfloor \log_{12}{\frac{frequency}{440}} \rceil\]
where $n=0$ for the \textit{440-Hz-A}, $n=1$ for \textit{A$\sharp$}, $n=-9$ for the \textit{middle C}, etc.

The \textit{note\_name} and \textit{octave} are then calculated simply:
\[note\_name = NOTES[n \bmod 12]\]
where
\[NOTES = [A, As, B, C, Cs, D, Ds, E, F, Fs, G, Gs]\]
and
\[octave = 2 + \lfloor \frac{n}{12} \rfloor\]
where $2$ is an octave of the \textit{440-Hz-A}.

Only the maximum volume value is selected from the frequencies that are converted to the same note.

\subsection{Event detection}\label{subsec:event-detection}
\textit{EventDetector} from \textit{event\_detection} module goes through the spectrogram (time-volume representation of
the notes) of the sound with overlapping window as shown in \cref{fig:pitch-detection} (green). The size of the window
is determined by 3 parameters: shortest note value (default is 16(\Sech)), sampling rate of input stream, and
overlapping rate which defines how many data points (volume in time) are shared between two consequent windows. Having
these 3 parameters, the size of the window is calculated as
\[window\_size = sampling\_rate / (shortest\_note\_value / overlapping\_rate)\]
and step between windows as
\[window\_step = \lfloor{sampling\_rate / shortest\_note\_value}\rfloor\]

The window of spectrogram is passed to the pretrained model for the given instrument to detect whether this window
contains the envelope(s) for the note(s) in this point in time. If it does, this window constitutes the start of
the note. As was mention in \cref{sec:event-detection}, the end of the detected note is a point in time when its volume
drops under specified threshold $note\_volume-10$.



\chapter{Testing}\label{ch:testing}

\section{Source separation}\label{sec:testing:source-separation}
Commonly used source separation quality evaluation metrics are presented in a paper by Emmanuel Vincent et al.~\cite{vincent2006performance}.
For a separating performance measures are computed for each estimated source $\hat{s}_j$ by comparing it to a given true
source $s_j$. The computation of the criteria involves two successive steps. In a first step, they decompose $\hat{s}_j$
as:
\[\hat{s}_j = s_{target} + e_{interf} + e_{noise} + e_{artif}\]
where $s_{target} = f(s_j)$ is a version of $s_j$ modified by an allowed distortion $f \in F$, and where $e_{interf}$,
$e_{noise}$ and $e_{artif}$ are the interferences, noise, and artifacts error terms respectively. These four terms
represent the parts of $\hat{s}_j$ that come from the real source $s_j$, from unwanted sources $(s_i)_{i \neq j}$, from
noise, and from other causes. The performance of the model then is evaluated by 4 metrics:

\begin{itemize}
	\item \ac{SDR} = $10 \log_{10} \frac{\parallel s_{target} \parallel^2}{\parallel e_{interf} + e_{noise} + e_{artif} \parallel^2}$
	\item \ac{SIR} = $10 \log_{10} \frac{\parallel s_{target} \parallel^2}{\parallel e_{interf} \parallel^2}$
	\item \ac{SNR} = $10 \log_{10} \frac{\parallel s_{target} + e_{interf} \parallel^2}{\parallel e_{noise} \parallel^2}$
	\item \ac{SAR} = $10 \log_{10} \frac{\parallel s_{target} + e_{interf} + e_{noise} \parallel^2}{\parallel e_{artif} \parallel^2}$
\end{itemize}

The Spleeter's performance measured on the standard musdb18 dataset~\cite{musdb18} comparing to \textit{Open-Unmix}~\cite{stoter19}
implementation is shown in \cref{fig:spleeter-performance}.

\begin{table}[]
	\begin{center}
		\begin{tabular}{l|cccc|cccc}
			\hline
			& SDR & SIR & SAR & ISR & SDR & SIR & SAR & ISR           \\ \hline \hline
			& \multicolumn{4}{c|}{vocals}                                     & \multicolumn{4}{c}{bass}                                      \\ \hline
			Spleeter & \textbf{6.88} & \textbf{15.86} & \textbf{6.99} & \textbf{12.01} & \textbf{5.51} & 10.30 & 5.96 & \textbf{9.61} \\
			Open-Unmix & 6.32 & 13.33 & 6.52 & 11.93 & 5.23 & \textbf{10.93} & \textbf{6.34} & 9.23          \\ \hline \hline
			& \multicolumn{4}{c|}{drums}                                     & \multicolumn{4}{c}{other}                                      \\ \hline
			Spleeter & \textbf{6.71} & \textbf{13.67} & \textbf{6.54} & \textbf{10.69} & \textbf{4.55} & \textbf{8.16} & \textbf{4.88} & \textbf{9.87}  \\
			Open-Unmix & 5.73 & 11.12 & 6.02 & 10.51 & 4.02 & 6.59 & 4.74 & 9.31           \\ \hline
		\end{tabular}
		\caption{Spleeter and Open-Unmix performances.}
		\label{fig:spleeter-performance}
	\end{center}
\end{table}


\section{Sound envelope detection}\label{sec:sound-envelope-detection}

Sound envelope detection models were trained and tested on the \textit{nsynth} dataset~\cite{nsynth2017}. For
the purposes of the model selection, the dataset generated from the \textit{nsynth} data was split into four cross
validation folds. Performance results averaged among the folds are shown in \cref{fig:envelope-detection-performance}.

\begin{table}[H]
	\begin{center}
		\hspace*{-1cm}\begin{tabular}{l|cc|ccc|ccc}
			\hline
								& \multicolumn{2}{c|}{time (s)} & \multicolumn{3}{c|}{test (\%)} & \multicolumn{3}{c}{train (\%)} \\
	        model 				& fit & score & accuracy & precision & recall & accuracy & precision & recall \\ \hline \hline
			MLP		 			& 6.22 & 0.05 & \textbf{65.0} & \textbf{70.8} & 57.3 & 67.3 & 72.4 & 60.6 \\
			RandomForest 		& 3.72 & 0.48 & 64.6 & 66.6 & \textbf{61.7} & 1.00 & 1.00 & 1.00 \\
			XGB 				& 2.45 & 0.08 & 64.3 & 67.0 & 60.3 & 67.1 & 70.1 & 63.2 \\
			AdaBoost 			& 1.72 & 0.32 & 63.3 & 65.5 & 60.0 & 65.7 & 68.4 & 62.0 \\
			DecisionTree 		& 0.25 & 0.03 & 59.4 & 59.4 & 59.4 & 59.9 & 59.8 & 60.0 \\
			Logistic			& 0.12 & 0.02 & 58.0 & 57.6 & 59.0 & 1.00 & 1.00 & 1.00 \\
			GaussianNB 			& 0.01 & 0.02 & 57.0 & 59.6 & 52.6 & 57.9 & 60.3 & 53.4 \\
			KNeighbors 			& 0.01 & 0.93 & 51.2 & 52.2 & 48.2 & 60.3 & 61.9 & 57.3 \\ \hline
		\end{tabular}
		\caption{Envelope detection models' performance.}
		\label{fig:envelope-detection-performance}
	\end{center}
\end{table}

As seen from \cref{fig:envelope-detection-performance}, the best performing model is \textit{multilayer perceptron
classifier}. It is a neural network with two hidden layers with 64 and 32 neurons respectively. It uses L2
regularization with $\alpha=0.001$ and adam optimizer for learning.

The accuracy of 65\% is not a competitive performance, so this part requires some tuning of hyperparameters, network
topology, or a completely different approach.

\subsection{Other modules}\label{subsec:other-modules}
The \textit{key} estimation module does not need a testing dataset to evaluate its performance. It is simple rule
based solution. Its accuracy is defined by ratio of musical pieces (say $k$) that follow its key - have more sharp notes
for notes that have sharp symbol in the key signature and the same for flat and natural notes; and ratio of usage of
major keys (as a major key is assumed by default in the implementation) among scores (say $p$). Having those two ratios
the accuracy of this solution is $k*p$. Although the choice of the major or corresponding to it minor key does not
affect the key signature, so accuracy for key signature estimation in output sheet music is just $k$.

\textit{Madmom} does not provide evaluation of performance of \textit{tempo} estimation functionality. So it has been
tested. The \ac{FMA} dataset~\cite{fma_dataset} was used for evaluation of \textit{madmom}'s tempo prediction. Even
though tempo estimation seems like a regression problem, small inaccuracies of the predictions are equally as bad as
big ones as they deteriorate the whole subsequent analysis. So the metric for the problem would be the accuracy that
shows a ratio of correctly estimated tempos to the number of analysed songs. On the small version \ac{FMA} dataset with
8000 30s songs, the accuracy of \textit{madmom}'s tempo estimation was 81.3\%.

\setsecnumdepth{part}
\chapter{Conclusion}\label{ch:conclusion}


\section{Possible improvements}

Chords estimation, crescendo - descendo

Key note probability

Multiresolution FFT
https://pdfs.semanticscholar.org/d55f/984d569786e1bbf945f7683361ffbbfff79a.pdf



\bibliographystyle{iso690}
\bibliography{bibliography}

\setsecnumdepth{all}
\appendix


\chapter{Acronyms}\label{ch:acronyms}
\begin{acronym}
	\acro{WAVE}{Waveform Audio File Format}
	\acro{LPCM}{Linear pulse-code modulation}
	\acro{FFT}{Fast Fourier transform}
	\acro{DFT}{Discrete Fourier transform}
	\acro{IDFT}{Inverse Discrete Fourier transform}
	\acro{CNN}{Convolutional Neural Network}
	\acro{SiSeC}{Source Separation campaign}
	\acro{AMT}{Automatic music transcription}
	\acro{ML}{Machine learning}
	\acro{MIREX}{Music Information Retrieval Evaluation eXchange}
	\acro{HPS}{Harmonic partial sequence}
	\acro{MAP}{Maximum a posteriori}
	\acro{EM}{Expectation-maximisation}
	\acro{HTC}{harmonic temporal structured clustering}
	\acro{NMF}{Non-negative matrix factorisation}
	\acro{SAGE}{Space alternating generalised \ac{EM} algorithm}
	\acro{HMM}{Hidden Markov model}
	\acrodefplural{HMM}{Hidden Markov models}
	\acro{DBN}{dynamic Bayesian network}
	\acrodefplural{DBN}{dynamic Bayesian networks}
	\acro{ADSR}{attack, decay, sustain, and release}
	\acro{MIDI}{Musical Instrument Digital Interface}

\end{acronym}


\chapter{Musical notation}\label{ch:music_notation}

Music notation, when properly applied, can completely describe any musical score in a simple, concise manner. In order
to achieve this, music notation must describe all definable parameters of each sound, specifically\cite{mcgrain1990music}:

\begin{itemize}
	\item duration
	\item pitch
	\item dynamic
	\item timbre
\end{itemize}

\textbf{Duration} is described by time signature ($4\atop4$, $3\atop4$, $7\atop8$, etc.), tempo (primarily, beats per
minute: \Vier~=~120), and duration values of note-heads (\Ganz,~\Halb,~\Vier,~\AAcht,~\Sech,~etc.) and rests
(\GaPa,~\ViPa,~\AcPa,~\SePa,~etc.):

\figcenter\lilypondfile{ly/music_notation/duration.ly}


\textbf{Pitch} is defined by position of the note on the staff, key, accidentals ($\flat$,~$\sharp$,~$\natural$) and
the specified clef (\inlinemusic\smalltrebleclef, \inlinemusic\smallbassclef, \inlinemusic\smallaltoclef, etc.):

\figcenter\lilypondfile{ly/music_notation/pitch.ly}


\textbf{Dynamic} of a sound describes its amplitude or loudness (\pp, \p, \mf, \f, \ff, etc.), its emotional intensity
and change through time.

\textbf{Timbre} describes specific color of a played note/sound. Timber primarily depends on the instrument played but
also can define other instrumental directions (i.e. \textit{on bell of cymbal}, etc.)


% The Staff
\section{The Staff}\label{sec:the_staff}
The base for all musical scores is the \textit{staff}. All other music symbols go are placed on the staff or in relation
to it.

The staff consists of five horizontal lines and four spaces between the lines. Every note-head is placed on one of
the lines or on one of the spaces between the lines. The higher the note-head on the staff - the higher the pitch of
the produced note.

\figcenter\lilypondfile{ly/music_notation/staff.ly}


\section{Leger Lines}\label{sec:leger_lines}
Obviously, five lines and five spaces can provide only limited range of notes (precisely, eleven places to put
the note-head, including just beneath the first(bottom) line and above fifth(top) line). If notes from outside this
range are needed, they are placed on or between so-called \textit{Leger lines}. These are the lines placed above or
beneath the main staff only in places where they are needed, so for each note individually.

\figcenter\lilypondfile{ly/music_notation/leger_lines.ly}

% Clefs
\section{Clefs}\label{sec:clefs}

The specified \textit{clef} defines location of each pitch on the staff. The most commonly used clefs are the Treble and
the Bass clefs\cite{an-explanation-of-clefs}.

\subsection{The Treble Clef}\label{subsec:the-treble-clef}

The \textit{Treble Clef} (or \textit{G clef}, because the middle curl of it encircles line on the staff that represents a
G-note) is used for most high-sounding instruments (i.e.\ violin, guitar, ukulele, flute, clarinet, saxophone, trumpet,
etc.).

\figcenter\lilypondfile{ly/music_notation/treble_clef.ly}

As it defines second line as G, the lines on the staff, from bottom to top, are E, G, B, D, F. The spaces then
are F, A, C, E\@. The middle C\footnote{\textit{Middle C} is a commonly used reference note. It is a closest C to
the middle of a standard 88 key piano (specifically, fourth C from the left). It is around 261.63 hertz.} goes on
the first leger line below the treble staff.

\subsection{The Bass Clef}\label{subsec:the-bass-clef}

The \textit{Base Clef} (or \textit{F clef}, because line between two dots on the symbol represents an F-note) is used for
low sounding instruments (i.e.\ bass guitar, cello, trombone, tuba, etc.)

\figcenter\lilypondfile{ly/music_notation/bass_clef.ly}

As it defines fourth line as F, the lines on the staff are G, B, D, F, A, and the spaces are A, C, E, G\@. The middle C
goes on the first leger line above the bass clef.


\subsection{The Percussion Clef}\label{subsec:the-percussion-clef}

The \textit{Percussion Clef} is commonly used for drum-set notation. Each line and space represent different part of
the drum kit. They are often predefined at the start of the part in so-called \textit{key} or \textit{legend}, or when
they first appear in the score.

\figcenter\lilypondfile{ly/music_notation/percussion_clef.ly}


\subsection{The Alto and Tenor Clefs}\label{subsec:the-alto-and-tenor-clefs}

\textit{Alto Clef} (or \textit{C clef}, because line in the middle of the alto staff represent middle C) and
The \textit{Tenor Clef} are less often used clefs. The viola and the alto trombone are generally the only instruments that
use the Alto clef. Tenor clef is occasionally used to represent the upper ranges of the cello, double bass, bassoon,
and trombone.

\figcenter\lilypondfile{ly/music_notation/alto_tenor_clefs.ly}

The lines of the alto staff are F, A, C, E, G, and the spaces are G, B, D, F\@. Similarly, for tenor clef, C is moved up
one line from alto clef, making the notes on the lines D, F, A, C, E and notes in the spaces E, G, B, D\@.

\subsection{The Great Staff}\label{subsec:the-great-staff}
The \textit{Great Staff} or the \textit{Grand Staff} is a combination of the treble staff and the bass staff. Usually
used by piano or harp musicians.

\figcenter\lilypondfile{ly/music_notation/great_staff.ly}

Often they also divide score into to parts played by left and right hand (i.e.\ on piano, treble clef part with
the right hand, bass clef part with left hand). So, even if some notes belong to treble clef they may be put on leger
lines above bass clef if played by left hand and vice versa.

\figcenter\lilypondfile{ly/music_notation/great_staff_leger_lines.ly}


\subsection{Key signature}\label{subsec:key-signature}
\textit{Key signature} is a series of sharp symbols or flat symbols placed on the staff, designating notes that are to
be consistently played one semitone higher or lower than the equivalent natural notes (for example, the white notes on
a piano keyboard) unless otherwise altered with an accidental. Key signatures are generally written immediately after
the clef at the beginning of a line of musical notation, although they can appear in other parts of a score, notably
after a double bar.\cite{key-signature}

\figcenter\lilypondfile{ly/music_notation/key_signature.ly}

Key \textit{D major} (defined in example above) consists of notes D, E, F$\sharp$, G, A, B,
C$\sharp$. So, after the clef, notes F and C marked with a $\sharp$, so, when they occur in
a score without any accidentals, they are played one semitone higher (C$\sharp$ instead of C, etc.)


% Rhythmic Description
\section{Rhythmic Description}\label{sec:rhythmic-description}
Alongside with pitch, it is required to describe rhythm. \textit{Rhythmic description} determines exactly when note
should be played and when it should stop playing. Notationally it is defined by note-heads, stems, flags, beams, rests,
and time signature.

\subsection{Note-heads, stems, flags, beams}\label{subsec:note-heads}

There are two types of note heads open and closed.
\figcenter\lilypondfile{ly/music_notation/note_heads.ly}

\textit{Stems} are vertical lines attached to the side of the notes-head. Together with flags, beams, and augmentation
dots they define duration value:

\figcenter\lilypondfile{ly/music_notation/notes_duration.ly}

Two half note have the same duration as one whole note, two quoter notes have the same duration as one half note and so
on.


\subsection{Rests}\label{subsec:rests}
Same as for notes, we can define pauses in music - \textit{rests}:

\figcenter\lilypondfile{ly/music_notation/rests_duration.ly}

Whole rest, half rest, quoter rest, and so on accordingly.

\subsection{Time signatures}\label{subsec:time-signatures}
Time signature is a sign that indicates the metre of a composition. Most time signatures consist of two vertically
aligned numbers, such as $2\atop2$, $3\atop4$, $6\atop8$, and $11\atop16$. The top figure reflects the number of beats
in each measure, or metrical unit; the bottom figure indicates the note value that receives one beat (here,
respectively, half note, quarter note, eighth note, and sixteenth note). When measures contain an uneven number of
beats falling regularly into two subgroups, the division may be indicated as, for instance, ${3+4}\atop4$ instead
of $7\atop4$\cite{time-signature}.

\figcenter\lilypondfile{ly/music_notation/time_signature.ly}

$4\atop4$ is such a common time signature that sometimes it is specified with \Takt{c}{0} and $2\atop2$ as \Takt{c}{1}.



\chapter{Contents of enclosed CD}\label{ch:contents-of-enclosed-cd}

%change appropriately

\begin{figure}
	\dirtree{%
		.1 readme.txt\DTcomment{the file with CD contents description}.
		.1 exe\DTcomment{the directory with executables}.
		.1 src\DTcomment{the directory of source codes}.
		.2 wbdcm\DTcomment{implementation sources}.
		.2 thesis\DTcomment{the directory of \LaTeX{} source codes of the thesis}.
		.1 text\DTcomment{the thesis text directory}.
		.2 thesis.pdf\DTcomment{the thesis text in PDF format}.
		.2 thesis.ps\DTcomment{the thesis text in PS format}.
	}
\end{figure}

\end{document}
